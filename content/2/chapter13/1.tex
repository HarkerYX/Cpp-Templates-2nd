C++用各种方法对名称进行分类——方法非常多。为了帮助处理如此丰富的术语,我们提供了表13.1和表13.2,其中描述了这些分类。幸运的是,通过熟悉两个主要的命名概念,就可以很好地了解大多数C++模板问题:

\begin{enumerate}
\item 
如果名称所属的作用域是使用范围解析操作符(::)或成员访问操作符()显式表示的,则该名称为限定名称。或\texttt{->})。例如,this\texttt{->}count是限定名,但count不是(即使普通计数实际上可能引用类成员)。

\item 
如果名称在某种程度上依赖于模板参数,那么它就是从属名称。例如,如果T是模板参数,那么std::vector<T>::iterator通常是一个依赖名称,但如果T是一个已知的类型别名(例如使用T= int时的T),那么它就是一个非依赖名称。
\end{enumerate}

通读这些表以熟悉用来描述C++模板问题的术语,但没必要记住每个术语的确切含义。如果需要,可以很容易地在索引中找到它们。

\begin{table}[H]
	\centering
	\begin{tabular}{|l|l|}
		\hline
		\textbf{分类} & \textbf{解释及注意事项}                                                                                                                                                                                                                                                                                                                                                                                                                                                                                                            \\ \hline
		\begin{tabular}[c]{@{}l@{}}标识符\\ Identifier  \end{tabular}            & \begin{tabular}[c]{@{}l@{}}只能由不间断的字母、下划线和数字组成的名称。不能以数字开头,并且有\\ 些标识符是保留的: 不应该在程序中引入(根据经验,避免开头下划线和双下\\ 划线)。“字母”的概念应该得到广泛的理解,包括从非字母语言编码字形的\\ 特殊通用字符名称(UCNs, universal character names)。\end{tabular}      \\ \hline
		\begin{tabular}[c]{@{}l@{}}操作符函数标识\\ Operator-function-id  \end{tabular}  & \begin{tabular}[c]{@{}l@{}}关键字operator后面跟着运算符的符号——例如,operator new和operator {[} {]}。\end{tabular}                                                                                                                                                                                                                                                                                                                                                                                     \\ \hline
		\end{tabular}
\end{table}

\begin{center}
表13.2. 名称的分类 
\end{center}

\begin{table}[H]
\centering
\begin{tabular}{|l|l|}
		\hline
		\begin{tabular}[c]{@{}l@{}}转换函数标识\\ Conversion-function-id \end{tabular} & \begin{tabular}[c]{@{}l@{}}用于表示用户定义的隐式转换操作符——例如,操作符int\&,很容易与\\ 位与操作(bitand)混淆与。\end{tabular}                                                                                                                                                                                                                                                                                                                                                \\ \hline
		\begin{tabular}[c]{@{}l@{}}字面操作标识\\ Literal-operator-id  \end{tabular}   & \begin{tabular}[c]{@{}l@{}}用于表示用户定义的文字操作符例如——操作符""\_km,可用来表示字\\ 面值(比如100\_km)(C++11).\end{tabular}                                                                                                                                                                                                                                                                                                                                 \\ \hline
		\begin{tabular}[c]{@{}l@{}}模板标识\\ Template-id   \end{tabular}         & \begin{tabular}[c]{@{}l@{}}模板的名称,后面跟着用尖括号括起来的模板参数,例如:List<T, int, 0>。\\ 也可以是一个操作符函数标识或字面操作标识,后面加上用尖括号括起\\ 来的模板参数; 例如:operator+<X<int\textgreater{}\textgreater{}\end{tabular}                                                                                                                               \\ \hline
		\begin{tabular}[c]{@{}l@{}}非限定性标识\\ Unqualified-id    \end{tabular}      & \begin{tabular}[c]{@{}l@{}}标识符的泛化,可以是上面的标识符(标识符,操作符函数标识,转换\\ 函数标识符,转换函数标识,字面操作标识或模板标识)或是“析构\\ 函数名称”(例如,像$\sim$Data 或 $\sim$List\textless{}T, T, N\textgreater{})。\end{tabular}                                                                                                                                                                                                                  \\ \hline
		\begin{tabular}[c]{@{}l@{}}限定性标识\\ Qualified-id    \end{tabular}        & \begin{tabular}[c]{@{}l@{}}一个非限定标识由类、枚举或名称空间的名称限定,或仅由全局作用\\ 域解析操作符限定。注意,这样的名称本身可以是限定的。 例如 ::X, S::x,\\ Array\textless{}T\textgreater{}::y和::N::A\textless{}T\textgreater{}::z.\end{tabular}                                                                                                                                                                                                    \\ \hline
		\begin{tabular}[c]{@{}l@{}}限定名\\ Qualified name    \end{tabular}      & \begin{tabular}[c]{@{}l@{}}这个术语在标准中没有定义,我们用它来指代经过限定查找的名称。具体\\ 来说,这是一个限定性标识或一个非限定标识,用于显式成员访问操作符\\ (.或-\textgreater{})。例如 S::x, this-\textgreater{}f, and p-\textgreater{}A::m。但是,在隐式等效于此的上下文\\ 中只使用class\_mem等价于this-\textgreater{}class\_mem不是限定名:成员访问必须显\\ 式。\end{tabular} \\ \hline
		\begin{tabular}[c]{@{}l@{}}非限定名称\\  Unqualified name    \end{tabular}  & \begin{tabular}[c]{@{}l@{}}非限定标识不是限定名称。这不是一个标准术语,而对应于标准所称的非\\ 限定名称的查找。\end{tabular}                                                                                                                                                                                                                                                                                                                                 \\ \hline
		\begin{tabular}[c]{@{}l@{}}名称 Name         \end{tabular}           & 一个限定或不限定的名字。                                                                                                                                                                                                                                                                                                                                                                                                                                                                                              \\ \hline
		\begin{tabular}[c]{@{}l@{}}相关名称\\ Dependent name \end{tabular}   & \begin{tabular}[c]{@{}l@{}}在某种程度上依赖于模板参数的名称。通常,显式包含模板形参的限定\\ 名或非限定名是依赖的。此外,成员访问操作符(. 或 \texttt{-\textgreater{}})通常是依赖于\\ 访问操作符左侧表达式的类型,这个概念在13.3.6节中讨论过。特别地,\\ 当b出现在模板中时(this\texttt{-\textgreater{}}b),它通常是一个依赖名称。最后,需要根\\ 据参数进行查找的名称(见第217页13.2节),例如调用ident(x, y)形式\\ 的ident或表达式x + y中的+中的ident,当且仅当任何参数表达式是类\\ 型依赖的时候,它就是一个依赖名称。\end{tabular} \\ \hline
		\begin{tabular}[c]{@{}l@{}}不相关名称\\ Nondependent name\end{tabular} & 不属于上述描述的名称。                                                                                                                                                                                                                                                                                                                                                                                                                                                                                                                                                                                                                                                                                                                                                                                                                   \\ \hline
	\end{tabular}
\end{table}





































