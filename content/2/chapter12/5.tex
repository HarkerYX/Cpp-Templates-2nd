
友元声明的基本思想很简单:标识与出现友元声明的类有特权连接的类或函数。然而,由于两个原因,事情有些复杂:

\begin{enumerate}
\item 
友元声明可能是实体的唯一声明。

\item 
友元函数声明可以是定义。
\end{enumerate}

\subsubsubsection{12.5.1\hspace{0.2cm}类模板的友元类}

友元类声明不能是定义,因此很少有问题。在模板环境中,友元类声明的唯一方式是能够将一个特定的类模板实例命名为友元:

\begin{lstlisting}[style=styleCXX]
template<typename T>
class Node;

template<typename T>
class Tree {
	friend class Node<T>;
	...
};
\end{lstlisting}

注意,当类模板的实例成为类或类模板的友元时,类模板必须可见。对于普通类,则没有这样的要求:

\begin{lstlisting}[style=styleCXX]
template<typename T>
class Tree {
	friend class Factory; // OK even if first declaration of Factory
	friend class Node<T>; // error if Node isn’t visible
};
\end{lstlisting}

13.2.2节对此有更多的说明。

在5.5节中介绍的一个应用是声明其他类模板实例化为友元:

\begin{lstlisting}[style=styleCXX]
template<typename T>
class Stack {
	public:
	...
	// assign stack of elements of type T2
	template<typename T2>
	Stack<T>& operator= (Stack<T2> const&);
	// to get access to private members of Stack<T2> for any type T2:
	template<typename> friend class Stack;
	...
};
\end{lstlisting}

C++11添加了使模板参数成为友元的语法:

\begin{lstlisting}[style=styleCXX]
template<typename T>
class Wrap {
	friend T;
	...
};
\end{lstlisting}

这对任何类型T都有效,但如果T不是实际的类类型,则会忽略。

\subsubsubsection{12.5.2\hspace{0.2cm}类模板的友元函数}

函数模板的实例可以成为友元函数,方法是确保友元函数的名称后面加上尖括号。尖括号可以包含模板参数,若参数可以推导,则尖括号可以为空:

\begin{lstlisting}[style=styleCXX]
template<typename T1, typename T2>
void combine(T1, T2);

class Mixer {
	friend void combine<>(int&, int&);
	// OK: T1 = int&, T2 = int&
	friend void combine<int, int>(int, int);
	// OK: T1 = int, T2 = int
	friend void combine<char>(char, int);
	// OK: T1 = char T2 = int
	friend void combine<char>(char&, int);
	// ERROR: doesn’t match combine() template
	friend void combine<>(long, long) { ... }
	// ERROR: definition not allowed!
};
\end{lstlisting}

注意,不能定义模板实例(最多只能定义特化),因此命名实例的友元声明不能成为定义。

如果名称后面没有尖括号,则有两种可能:

\begin{enumerate}
\item 
如果名称没有限定(不包含::),永远不会引用模板实例。如果在友元声明处没有匹配的非模板函数可见,则友元声明是该函数的第一个声明。声明也可以是定义。

\item 
如果该名称是限定的(包含::),则该名称必须引用先前声明的函数或函数模板。匹配的函数优先于匹配的函数模板。但是,这样的友元声明不能是定义。
\end{enumerate}

举一个例子可以帮助阐明各种可能性:

\begin{lstlisting}[style=styleCXX]
void multiply(void*); // ordinary function

template<typename T>
void multiply(T); // function template

class Comrades {
	friend void multiply(int) { }
	// defines a new function ::multiply(int)
	
	friend void ::multiply(void*);
	// refers to the ordinary function above,
	// not to the multiply<void*> instance
	
	friend void ::multiply(int);
	// refers to an instance of the template
	
	friend void ::multiply<double*>(double*);
	// qualified names can also have angle brackets,
	// but a template must be visible
	
	friend void ::error() { }
	// ERROR: a qualified friend cannot be a definition
};
\end{lstlisting}

前面的例子中,在一个普通类中声明了友元函数。同样的规则也适用于在类模板中的声明,但是模板参数可以参与识别友元函数:

\begin{lstlisting}[style=styleCXX]
template<typename T>
class Node {
	Node<T>* allocate();
	...
};

template<typename T>
class List {
	friend Node<T>* Node<T>::allocate();
	...
};
\end{lstlisting}

友元函数也可以在类模板中定义,其只在实际使用时实例化。这通常要求友元函数在类型中使用类模板本身,这使得在类模板上更容易表示,可以像在命名空间中调用函数一样:

\begin{lstlisting}[style=styleCXX]
template<typename T>
class Creator {
	friend void feed(Creator<T>) { // every T instantiates a different function ::feed()
		...
	}
};

int main()
{
	Creator<void> one;
	feed(one); // instantiates ::feed(Creator<void>)
	Creator<double> two;
	feed(two); // instantiates ::feed(Creator<double>)
}
\end{lstlisting}

这个例子中,每次Creator的实例化都会生成一个不同的函数。请注意,尽管这些函数是作为模板实例化的一部分生成的,但这些函数本身是普通函数,而不是模板的实例。但是,它们是模板实体(参见12.1节),它们的定义只有在使用时才会实例化。还要注意,因为这些函数体是在类定义内部定义的,所以是隐式内联的。因此,在两个不同的编译单元中生成相同的函数不会出错。13.2.2节和21.2.1节对这个主题有更多的说明。

\subsubsubsection{12.5.3\hspace{0.2cm}友元模板}

通常,当声明函数或类模板实例的友元时,可以明确表示哪个实体是友元。尽管如此,有时表示模板的所有实例都是类的友元也是有用的。这需要一个友元模板:

\begin{lstlisting}[style=styleCXX]
class Manager {
	template<typename T>
	friend class Task;
	
	template<typename T>
	friend void Schedule<T>::dispatch(Task<T>*);
	
	template<typename T>
		friend int ticket() {
			return ++Manager::counter;
		}
	static int counter;
};
\end{lstlisting}

就像普通的友元声明一样,友元模板只有在命名了一个非限定的、后面没有尖括号的函数名时才可以定义。

友元模板只能声明主模板和主模板的成员。与主模板关联的偏特化和显式特化也认为是友元。








