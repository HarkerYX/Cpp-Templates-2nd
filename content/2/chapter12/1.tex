
C++目前支持四种基本模板:类模板、函数模板、变量模板和别名模板。这些模板类型都可以出现在名称空间中,也可以出现在类中。在类作用域中,作为嵌套的类模板、成员函数模板、静态数据成员模板和成员别名模板。这些模板的声明与普通的类、函数、变量和类型别名(或类成员)相似,不过需要参数化

\begin{lstlisting}[style=styleCXX]
template<parameters here>
\end{lstlisting}

C++17引入了通过参数化子句引入的构造:推导策略(请参阅2.9节和15.12.1节)。本书中,它们不叫做模板(例如,没有实例化),但是选择这个语法是为了让人联想到函数模板。

将在后面的章节中回到模板参数声明。首先,使用一些示例展示这四种模板,其可以出现在命名空间中(全局或在一个命名空间中):

\hspace*{\fill} \\ %插入空行
\noindent
\textit{details/definitions1.hpp}
\begin{lstlisting}[style=styleCXX]
template<typename T> // a namespace scope class template
class Data {
	public:
	static constexpr bool copyable = true;
	...
};

template<typename T> // a namespace scope function template
void log (T x) {
	...
}

template<typename T> // a namespace scope variable template (since C++14)
T zero = 0;

template<typename T> // a namespace scope variable template (since C++14)
bool dataCopyable = Data<T>::copyable;

template<typename T> // a namespace scope alias template
using DataList = Data<T*>;
\end{lstlisting}

本例中,尽管是通过类模板Data的参数化间接参数化的,但静态数据成员Data<T>::copyable不是变量模板。变量模板可以出现在类中,所以其是一个静态数据成员模板。

下面的例子展示了,四种模板作为成员,在如何在父类中进行定义:

\hspace*{\fill} \\ %插入空行
\noindent
\textit{details/definitions2.hpp}
\begin{lstlisting}[style=styleCXX]
class Collection {
	public:
	template<typename T> // an in-class member class template definition
	class Node {
		...
	};

	template<typename T> // an in-class (and therefore implicitly inline)
	T* alloc() { // member function template definition
		...
	}

	template<typename T> // a member variable template (since C++14)
	static T zero = 0;
	
	template<typename T> // a member alias template
	using NodePtr = Node<T>*;
};
\end{lstlisting}

C++17中,变量(包括静态数据成员)和变量模板可以“内联”,定义可以跨翻译单元进行重复。这对于变量模板来说是多余的,因为变量模板可以定义在多个翻译单元中。但与成员函数不同,在封闭类中定义的静态数据成员不会使其内联:必须使用关键字inline指定。

最后,下面的代码演示了如何在类外定义非别名模板的成员模板:

\hspace*{\fill} \\ %插入空行
\noindent
\textit{details/definitions3.hpp}
\begin{lstlisting}[style=styleCXX]
template<typename T> // a namespace scope class template
class List {
	public:
	List() = default; // because a template constructor is defined
	
	template<typename U> // another member class template,
	class Handle; // without its definition
	
	template<typename U> // a member function template
	List (List<U> const&); // (constructor)
	
	template<typename U> // a member variable template (since C++14)
	static U zero;
};

template<typename T> // out-of-class member class template definition
  template<typename U>
class List<T>::Handle {
	...
};

template<typename T> // out-of-class member function template definition
  template<typename T2>
List<T>::List (List<T2> const& b)
{
	...
}

template<typename T> // out-of-class static data member template definition
  template<typename U>
U List<T>::zero = 0;
\end{lstlisting}

定义在类外的成员模板需要多个template<...>参数化子句:每个外围作用域的类模板一个,成员模板本身也需要一个,子句从类模板最外层开始逐行展示。

构造函数模板(一种特殊的成员函数模板)禁用默认构造函数的隐式声明(因为只有在没有声明其他构造函数的情况下,才会隐式声明)。添加默认声明

\begin{lstlisting}[style=styleCXX]
List() = default;
\end{lstlisting}

确保List<T>的实例是默认可构造的,具有隐式声明构造函数的语义。

\hspace*{\fill} \\ %插入空行
\noindent
\textbf{联合模板}

联合模板也可用(一种类模板):

\begin{lstlisting}[style=styleCXX]
template<typename T>
union AllocChunk {
	T object;
	unsigned char bytes[sizeof(T)];
};
\end{lstlisting}

\hspace*{\fill} \\ %插入空行
\noindent
\textbf{默认参数}

函数模板可以像普通函数声明一样有默认的参数:

\begin{lstlisting}[style=styleCXX]
template<typename T>
void report_top (Stack<T> const&, int number = 10);
template<typename T>

void fill (Array<T>&, T const& = T{}); // T{} is zero for built-in types
\end{lstlisting}

后面的声明表明默认参数可以依赖于模板参数,也可以定义为(C++11前唯一的方法,参见5.2节)

\begin{lstlisting}[style=styleCXX]
template<typename T>
void fill (Array<T>&, T const& = T()); // T() is zero for built-in types
\end{lstlisting}

当使用fill()时,若提供了第二个函数调用参数,则不会实例化默认参数。这确保了默认参数不能为特定的T进行实例化,从而不会出现错误。例如:

\begin{lstlisting}[style=styleCXX]
class Value {
public:
	explicit Value(int); // no default constructor
};

void init (Array<Value>& array)
{
	Value zero(0);
	fill(array, zero); // OK: default constructor not used
	fill(array); // ERROR: undefined default constructor for Value is used
}
\end{lstlisting}

\hspace*{\fill} \\ %插入空行
\noindent
\textbf{类模板的非模板成员}

除了在类内部声明的四种基本模板之外,还可以通过成为类模板的一部分来将普通类成员参数化,偶尔(错误地)也称为成员模板。尽管可以进行参数化,但这样的定义并不是模板,其参数完全由所属的模板决定。例如:

\begin{lstlisting}[style=styleCXX]
template<int I>
class CupBoard
{
	class Shelf; // ordinary class in class template
	void open(); // ordinary function in class template
	enum Wood : unsigned char; // ordinary enumeration type in class template
	static double totalWeight; // ordinary static data member in class template
};
\end{lstlisting}

因为不是模板,所以对应的定义只为父类模板指定参数化子句(参数化子句与出现在最后一个::之后的名称相关联):

\begin{lstlisting}[style=styleCXX]
emplate<int I> // definition of ordinary class in class template
class CupBoard<I>::Shelf {
	...
};

template<int I> // definition of ordinary function in class template
void CupBoard<I>::open()
{
	...
}

template<int I> // definition of ordinary enumeration type class in class template
enum CupBoard<I>::Wood {
	Maple, Cherry, Oak
};

template<int I> // definition of ordinary static member in class template
double CupBoard<I>::totalWeight = 0.0;
\end{lstlisting}

C++17后,静态的totalWeight成员可以在类模板中使用inline来初始化:

\begin{lstlisting}[style=styleCXX]
template<int I>
class CupBoard
	...
	inline static double totalWeight = 0.0;
};
\end{lstlisting}

尽管这种参数化的定义通常称为模板,但这个术语并不适用于。对这些实体的一个术语是temploid。C++17起,标准确实定义了模板化实体的概念,包括模板和temploid,以及递归地在模板化实体中定义或创建的实体(例如,包括在类模板中定义的友元函数(参见2.4节)或模板中出现的Lambda表达式的闭包类型)。目前,temploid和模板实体都没有获得太多的关注,但将来可以使用这些使用这些术语,在讨论与C++模板的话题时进行精确的表达。

\subsubsubsection{12.1.1\hspace{0.2cm}虚拟成员函数}

成员函数模板不能进行虚声明。这个约束是因为虚函数调用机制的实现,是使用一个固定大小的表,每个虚函数只有一个条目。但成员函数模板的实例化,直到整个程序翻译后才固定。因此,支持虚成员函数模板需要在C++编译器和链接器中支持一套全新的机制。

因为在实例化类时其数量固定,所以类模板的普通成员可以为虚函数:

\begin{lstlisting}[style=styleCXX]
template<typename T>
class Dynamic {
	public:
	virtual ~Dynamic(); // OK: one destructor per instance of Dynamic<T>
	
	template<typename T2>
	virtual void copy (T2 const&);
						// ERROR: unknown number of instances of copy()
						// given an instance of Dynamic<T>
};
\end{lstlisting}

\subsubsubsection{12.1.2\hspace{0.2cm}链接模板}

每个模板都必须有一个名字,并且这个名字在其作用域内必须唯一(除了函数模板可以重载(参见第16章))。与类类型不同,类模板不能与其他类型的实体共享名称:

\begin{lstlisting}[style=styleCXX]
int C;
...
class C; // OK: class names and nonclass names are in a different “space”

int X;
...
template<typename T>
class X; // ERROR: conflict with variable X
struct S;
...
template<typename T>
class S; // ERROR: conflict with struct S
\end{lstlisting}

模板名称有链接,但不能有C链接存在。非标准名称链接的含义可能与实现有关(然而,目前还不清楚有哪个实现支持模板的非标准名称链接):

\begin{lstlisting}[style=styleCXX]
extern "C++" template<typename T>
void normal(); // this is the default: the linkage specification could be left out

extern "C" template<typename T>
void invalid(); // ERROR: templates cannot have C linkage

extern "Java" template<typename T>
void javaLink(); // nonstandard, but maybe some compiler will someday
				// support linkage compatible with Java generics
\end{lstlisting}

模板通常具有外部链接。唯一的例外是具有静态说明符的命名空间作用域函数模板、未命名命名空间的直接或间接成员模板(具有内部链接),以及未命名类的成员模板(没有链接)。例如:

\begin{lstlisting}[style=styleCXX]
template<typename T> // refers to the same entity as a declaration of the
void external(); // same name (and scope) in another file

template<typename T> // unrelated to a template with the same name in
static void internal(); // another file

template<typename T> // redeclaration of the previous declaration
static void internal();

namespace {
	template<typename> // also unrelated to a template with the same name
	void otherInternal(); // in another file, even one that similarly appears
} // in an unnamed namespace

namespace {
	template<typename> // redeclaration of the previous template declaration
	void otherInternal();
}

struct {
	template<typename T> void f(T) {} // no linkage: cannot be redeclared
} x;
\end{lstlisting}

注意,由于后一个成员模板没有链接,因为没有办法在类之外提供定义,所以必须在未命名的类中定义。

目前模板不能在函数作用域或局部类作用域内声明,但是泛型Lambda(请参阅15.10.6节)可以出现在局部作用域中,具有包含成员函数模板的相关闭包,这是一种局部成员函数模板。

模板实例链接就是模板链接。从上面声明的模板内部实例化的函数internal<void>()将进行内部链接。对于变量模板,有一个有趣的结果,看看下面的例子:

\begin{lstlisting}[style=styleCXX]
template<typename T> T zero = T{};
\end{lstlisting}

所有0的实例化都有外部链接,甚至像zero<int const>这样的东西都有。这可能是反直觉的

\begin{lstlisting}[style=styleCXX]
int const zero_int = int{};
\end{lstlisting}

具有内部链接,因为使用const类型声明。模板的所有实例化为

\begin{lstlisting}[style=styleCXX]
template<typename T> int const max_volume = 11;
\end{lstlisting}

有外部链接,所有这些实例化也有int const类型。

\subsubsubsection{12.1.3\hspace{0.2cm}主模板}

模板的普通声明为主模板,就不需要模板名称后加上尖括号中的模板参数:

\begin{lstlisting}[style=styleCXX]
template<typename T> class Box; // OK: primary template
template<typename T> class Box<T>; // ERROR: does not specialize

template<typename T> void translate(T); // OK: primary template
template<typename T> void translate<T>(T); // ERROR: not allowed for functions

template<typename T> constexpr T zero = T{}; // OK: primary template
template<typename T> constexpr T zero<T> = T{}; // ERROR: does not specialize
\end{lstlisting}

非主模板是在声明类或变量模板的偏特化时发生,这些将在第16章中讨论。并且函数模板必须是主模板(参见17.3节的讨论)。


















