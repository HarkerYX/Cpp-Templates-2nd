本章讨论两个相关但不同的问题:C++模板编译模型和C++模板实例化机制。

编译模型在程序转换的各个阶段决定模板的含义。特别地,它确定了模板中各种构造在实例化时的含义。名称查找是编译模型的重要组成部分。

标准C++只支持一种编译模型,即包含模型。然而,1998年和2003年的标准也支持模板编译的分离模型,允许在与实例化不同的翻译单元中编写模板定义。这些导出的模板只由爱迪生设计团队(EDG)实现过。

\begin{tcolorbox}[colback=webgreen!5!white,colframe=webgreen!75!black]
\hspace*{0.75cm}具有讽刺意味的是,将该技术添加到标准文档中时,EDG是最强烈的反对者。
\end{tcolorbox}

实现工作确定了:(1)实现C++模板的分离模型比预期的要困难得多,耗时也多;(2)分离模型的假设优势(比如:改进编译时间),由于模型的复杂性,并没有实现。随着2011年标准的开发接近尾声,并且其他实现者不打算支持该特性,C++标准委员会投票决定从该语言中删除导出模板的特性。对分离模型的细节感兴趣的读者,推荐本书的第一版([VandevoordeJosuttisTemplates1st]),其中描述了导出模板的具体行为。

实例化机制允许C++实现正确创建实例化的机制,这些机制可能会受到链接器和其他软件构建工具的限制。虽然实例化机制在不同的实现中不同(每个实现都有其优缺点),但通常不会对C++的编程产生重大影响。

C++11完成后不久,Walter Bright、Herb Sutter和Andrei Alexandrescu提出了一个与constexpr if类似的“静态if”特性(通过N3329)。然而,它是一个更普遍的特性,甚至可以出现在函数定义之外(Walter Bright是D编程语言的主要设计者和实现者,D语言也有类似的功能)。例如:

\begin{lstlisting}[style=styleCXX]
template<unsigned long N>
struct Fact {
	static if (N <= 1) {
		constexpr unsigned long value = 1;
	} else {
		constexpr unsigned long value = N*Fact<N-1>::value;
	}
};
\end{lstlisting}

本例中,类作用域声明是有条件的。然而,这种强大能力具有争议,一些委员会成员担心它会滥用,而另一些则不喜欢该提议使用的某些技术(例如,大括号没有引入范围,并且丢弃的分支根本没有解析)。

几年后,Ville Voutilainen带着提案(P0128)回来了,就是后来的constexpr if语句。经过了一些设计迭代(涉及到暂定的关键字static\_if和constexpr\_if),在Jens Maurer的帮助下,Ville最终将该提案引入语言(通过P0292r2)。



