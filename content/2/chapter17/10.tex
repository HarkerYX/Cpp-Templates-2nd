参数包是在C++11中引入的,但处理它们通常需要递归模板实例化技术。回想一下14.6节中讨论的代码大纲:

\begin{lstlisting}[style=styleCXX]
template<typename Head, typename... Remainder>
void f(Head&& h, Remainder&&... r) {
	doSomething(h);
	if constexpr (sizeof...(r) != 0) {
		// handle the remainder recursively (perfectly forwarding the arguments):
		f(r...);
	}
}
\end{lstlisting}

通过使用C++17编译时if语句的特性(请参阅第8.5节),这个示例变得更简单了,但仍然属于递归实例化技术,编译起来可能会很耗时。

委员会的几项提案试图在某种程度上简化这种状况。一个例子是引入了从包中挑选特定元素的符号。特别地,对于一个组合P,有人建议用P[N]来表示该组合中的元素N+1。同样,也有人建议将包装表示为“片”(例如,使用P.[b, e]符号)。

研究这些建议时,可以清楚地看到它们与上面讨论的反射元编程概念有一定的相关性。目前还不清楚是否会将特定的包选择机制添加到该C++中,或者是否会提供满足这一需求的元编程工具。




















































