按值传递参数时,原则上必须复制每个参数。因此,每个参数都成为所传递实参的副本。对于类,作为副本创建的对象通常由复制构造函数初始化。

调用复制构造函数的代价可能会很高。然而,即使在按值传递参数时,也有方法来避免昂贵的复制:编译器可能会优化复制对象的复制操作,并且通过使用移动语义,即使对复杂的对象也可以变得廉价。

来看一个实现的简单函数模板,参数是通过值传递的:

\begin{lstlisting}[style=styleCXX]
template<typename T>
void printV (T arg) {
	...
}
\end{lstlisting}

当为整数调用此函数模板时,结果代码为

\begin{lstlisting}[style=styleCXX]
void printV (int arg) {
	...
}
\end{lstlisting}

参数arg成为传入参数的副本,无论它是对象、文字还是函数返回的值。

若定义一个std::string类型,并为此调用函数模板:

\begin{lstlisting}[style=styleCXX]
std::string s = "hi";
printV(s);
\end{lstlisting}

模板形参T被实例化为std::string,得到

\begin{lstlisting}[style=styleCXX]
void printV (std::string arg)
{
	...
}
\end{lstlisting}

同样,在传递字符串时,arg变成了s的副本。这次的副本是由string类的复制构造函数创建的,这是一个昂贵的操作,因为这个复制操作创建了一个完整或“深”副本,以便该副本内部分配自己的内存来保存该值。

\begin{tcolorbox}[colback=webgreen!5!white,colframe=webgreen!75!black]
\hspace*{0.75cm}string类的实现本身可能进行了一些优化,以降低复制成本。一种是小字符串优化(SSO),只要值不太长,就直接使用对象内部的一些内存来保存值,而不分配内存。另一种是写时复制优化,只要源文件和副本都没有修改,就会使用与源文件相同的内存创建副本。但是,写时复制优化在多线程代码中有显著的缺陷。由于这个原因,自C++11起,标准字符串是禁止使用写时复制优化。
\end{tcolorbox}

但是,复制构造函数并不总调用。考虑以下代码:

\begin{lstlisting}[style=styleCXX]
std::string returnString();
std::string s = "hi";
printV(s); // copy constructor
printV(std::string("hi")); // copying usually optimized away (if not, move constructor)
printV(returnString()); // copying usually optimized away (if not, move constructor)
printV(std::move(s)); // move constructor
\end{lstlisting}

第一次调用中,传递了一个左值,使用了复制构造函数。然而,第二次和第三次调用中,当直接调用函数模板来获取prvalues(动态创建或由另一个函数返回的临时对象;参见附录B),编译器通常会优化传递参数,这样就不会调用复制构造函数了。注意,从C++17起,这种优化是必需的。C++17前,不能优化复制的编译器,至少必须尝试使用移动语义,这通常会使复制成本降低。最后一次调用中,当传递xvalue(一个现有的带有std::move()的非常量对象)时,不再需要s的值来强制使用移动构造函数。

因此,调用printV()的实现来声明按值传递的参数,通常只在传递左值(之前创建的对象,因为没有使用std::move()来传递它,所以在之后仍然可以使用)时才会很昂贵。不幸的是,这是一个常见的情况。原因是,在早期创建对象,并在稍后(经过一些修改)将其传递给其他函数的操作很常见。

\hspace*{\fill} \\ %插入空行
\noindent
\textbf{值传递的类型衰变}

还有一个必须提到的按值传递的属性:当按值传递参数时,类型会衰变。从而数组将转换为指针,并删除const和volatile等限定符(就像使用值作为使用auto声明的对象的初始化式一样):

\begin{tcolorbox}[colback=webgreen!5!white,colframe=webgreen!75!black]
\hspace*{0.75cm}术语衰变来自C语言,也适用于从函数到函数指针的类型转换(参见11.1.1节)。
\end{tcolorbox}

\begin{lstlisting}[style=styleCXX]
template<typename T>
void printV (T arg) {
	...
}

std::string const c = "hi";
printV(c); // c decays so that arg has type std::string

printV("hi"); // decays to pointer so that arg has type char const*

int arr[4];
printV(arr); // decays to pointer so that arg has type char const*
\end{lstlisting}

因此,当传递字符串字面值"hi"时,类型char const[3]衰变为char const*,从而成为T的推导类型。

\begin{lstlisting}[style=styleCXX]
void printV (char const* arg)
{
	...
}
\end{lstlisting}

这种行为有利有弊,并源自C语言。简化了对传递的字符串字面量的处理,但缺点是在printV()中,无法区分传递单个元素的指针和传递数组。因此,将在7.4节中将讨论如何处理字符串字面值和其他数组。




