isPrime()等编译时测试的一种应用是在编译时使用偏特化在不同实现之间进行选择。

例如,可以根据模板参数是否是素数来选择不同的实现:

\begin{lstlisting}[style=styleCXX]
// primary helper template:
template<int SZ, bool = isPrime(SZ)>
struct Helper;

// implementation if SZ is not a prime number:
template<int SZ>
struct Helper<SZ, false>
{
	...
};

// implementation if SZ is a prime number:
template<int SZ>
struct Helper<SZ, true>
{
	...
};

template<typename T, std::size_t SZ>
long foo (std::array<T,SZ> const& coll)
{
	Helper<SZ> h; // implementation depends on whether array has prime number as size
	...
}
\end{lstlisting}

这里,根据std::array<>参数的大小是否为素数,使用了Helper<>类的两种不同实现。这种偏特化的应用广泛适用于函数模板根据模板参数,选择不同的实现。

上面,使用了两个偏特化来实现两个可能的替代方案。相反,也可以对其中一个替代(默认)情况使用主模板,并对任何其他特殊情况使用偏特化版本:

\begin{lstlisting}[style=styleCXX]
// primary helper template (used if no specialization fits):
template<int SZ, bool = isPrime(SZ)>
struct Helper
{
	...
};

// special implementation if SZ is a prime number:
template<int SZ>
struct Helper<SZ, true>
{
	...
};
\end{lstlisting}

因为函数模板不支持偏特化,所以必须使用其他机制根据某些约束来更改函数实现。可供的选择包括:

\begin{itemize}
\item 
使用带有静态函数的类,

\item 
使用std::enable\_if,在第6.3节中介绍。

\item 
使用接下来介绍的SFINAE特性,或者

\item 
使用编译时if特性,该特性从C++17引入,将在下面的8.5节中介绍。
\end{itemize}

第20章讨论了基于约束选择函数实现的技术。






























































