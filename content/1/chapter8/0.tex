C++总是会有一些在编译时计算值的简单方法。模板增加了这方面的可能性,而语言的进一步发展也增强了这种工具。

从而可以决定是否使用某些模板代码,或者在不同的模板代码之间进行选择。但若所有必要的输入都可用,编译器可以在编译时计算控制流的结果。

事实上,C++可以通过多种特性来支持编译时编程:

\begin{itemize}
\item 
C++98前,模板已经提供了编译时计算的能力,包括使用循环和执行路径选择。(然而,有些人认为这是对模板特性的“滥用”,因为非直观的语法。)

\item 
使用偏特化,可以在编译时根据约束或要求在不同的类模板实现之间进行选择。

\item 
使用SFINAE原则,可以针对类型或约束在函数模板实现之间进行选择。

\item 
C++11和C++14中,通过使用“直观的”执行路径选择,以及自C++14后的大多数语句类型(包括for循环、switch语句等)的constexpr特性,编译时计算得到了更好的支持。

\item 
C++17引入了“编译时if”解除依赖于编译时条件或约束的语句,其甚至可以在模板外工作。
\end{itemize}

本章将介绍这些特性,特别关注模板的角色和上下文。


