
\begin{itemize}
\item 
模板中,通过将参数声明为转发引用(声明为模板参数名称后跟\&\&形成的类型)并在转发调用中使用std::forward<>(),就可以“完美”地转发参数了。

\item 
当使用完美转发成员函数模板时,可能会比预定义的用于复制或移动对象的特殊成员函数更匹配。

\item 
使用std::enable\_if<>,可以在编译时条件为false时禁用函数模板(当条件确定,模板将忽略)。

\item 
通过std::enable\_if<>,可以避免为单个参数调用的构造函数模板或赋值操作符模板,比隐式生成的特殊成员函数更好匹配的问题。

\item 
通过删除const volatile预定义的特殊成员函数,可以对特殊成员函数进行模板化(并应用enable\_if<>)。

\item 
概念允许对函数模板的需求使用更直观的语法。
\end{itemize}