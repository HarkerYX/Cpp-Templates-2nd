与前几章的栈实现不同,这里可以通过使用固定大小的元素数组来实现栈。这种方法的优点是避免了内存管理开销,但这为堆栈确定最佳容量带来了困难。指定的容量越小,堆栈越有可能快速填满。指定的容量越大,有可能会浪费内存。好的解决方案是让堆栈的用户指定数组的大小,作为堆栈的最大容量。

为此,可以将size定义为模板参数:

\hspace*{\fill} \\ %插入空行
\noindent
\textit{basics/stacknontype.hpp}
\begin{lstlisting}[style=styleCXX]
#include <array>
#include <cassert>

template<typename T, std::size_t Maxsize>
class Stack {
private:
	std::array<T,Maxsize> elems; // elements
	std::size_t numElems; // current number of elements
public:
	Stack(); // constructor
	void push(T const& elem); // push element
	void pop(); // pop element
	T const& top() const; // return top element
	bool empty() const { // return whether the stack is empty
		return numElems == 0;
	}
	std::size_t size() const { // return current number of elements
		return numElems;
	}
};

template<typename T, std::size_t Maxsize>
Stack<T,Maxsize>::Stack ()
: numElems(0) // start with no elements
{
	// nothing else to do
}

template<typename T, std::size_t Maxsize>
void Stack<T,Maxsize>::push (T const& elem)
{
	assert(numElems < Maxsize);
	elems[numElems] = elem; // append element
	++numElems; // increment number of elements
}

template<typename T, std::size_t Maxsize>
void Stack<T,Maxsize>::pop ()
{
	assert(!elems.empty());
	--numElems; // decrement number of elements
}

template<typename T, std::size_t Maxsize>
T const& Stack<T,Maxsize>::top () const
{
	assert(!elems.empty());
	return elems[numElems-1]; // return last element
}
\end{lstlisting}

第二个模板形参Maxsize是int类型,指定了堆栈元素内部数组的容量:

\begin{lstlisting}[style=styleCXX]
template<typename T, std::size_t Maxsize>
class Stack {
private:
	std::array<T,Maxsize> elems; // elements
	...
};
\end{lstlisting}

另外,在push()中,可以使用它来检查堆栈是否已满:

\begin{lstlisting}[style=styleCXX]
template<typename T, std::size_t Maxsize>
void Stack<T,Maxsize>::push (T const& elem)
{
	assert(numElems < Maxsize);
	elems[numElems] = elem; // append element
	++numElems; // increment number of elements
}
\end{lstlisting}

要使用这个类模板,必须同时指定元素类型和最大容量:

\hspace*{\fill} \\ %插入空行
\noindent
\textit{basics/stacknontype.cpp}
\begin{lstlisting}[style=styleCXX]
#include "stacknontype.hpp"
#include <iostream>
#include <string>

int main()
{
	Stack<int,20> int20Stack; // stack of up to 20 ints
	Stack<int,40> int40Stack; // stack of up to 40 ints
	Stack<std::string,40> stringStack; // stack of up to 40 strings
	
	// manipulate stack of up to 20 ints
	int20Stack.push(7);
	std::cout << int20Stack.top() << '\n';
	int20Stack.pop();
	
	// manipulate stack of up to 40 strings
	stringStack.push("hello");
	std::cout << stringStack.top() << '\n';
	stringStack.pop();
}
\end{lstlisting}

每个模板实例化都是独立的类型。因此,int20Stack和int40Stack是两种不同的类型。不能使用其中一个来代替另一个,也不能将其中一个分配给另一个。

同样,可以指定模板参数的默认值:

\begin{lstlisting}[style=styleCXX]
template<typename T = int, std::size_t Maxsize = 100>
class Stack {
	...
};
\end{lstlisting}

但从设计的角度来看,这样的写法在本例中可能不太合适。默认参数是正确的,但对于一般的堆栈类型来说,int类型和最大容量为100似乎都不直观。因此,最好显式地指定这两个值,以便在声明期间进行记录。





























