C++语言定义对各种实体的声明施加了一些约束。这些约束的总和称为单一定义规则或ODR。这个规则的细节有点复杂,并且涉及到很多不同的情况。后面的章节将说明在每个适用的上下文中产生的各种情况,可以在附录A中找到ODR的完整描述。现在,记住ODR的基础知识就足够了:

\begin{itemize}
\item 
普通(即,不是模板)非内联函数和成员函数,以及(非内联)全局变量和静态数据成员在整个程序中应该只定义一次。

\begin{tcolorbox}[colback=webgreen!5!white,colframe=webgreen!75!black]
\hspace*{0.75cm}C++17后,全局变量和静态变量,以及数据成员都可以定义为内联。这样就不需要在同一个翻译单元中定义它们。
\end{tcolorbox}

\item 
类类型(包括结构和联合)、模板(包括偏特化,但不包括全特化)以及内联函数和变量在每个翻译单元中最多定义一次,而且所有这些定义应该相同。
\end{itemize}

翻译单元是对源文件进行预处理的结果;也就是说,包含了由\#include指令命名并由宏展开生成的内容。

本书的其余部分中,可链接实体指的是以下任何一种:一个函数或成员函数,一个全局变量或一个静态数据成员,包括从模板生成的,对链接器可见的东西。




































