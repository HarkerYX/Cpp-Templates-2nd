

函数模板提供了适用于不同数据类型的行为,代表的是一组函数。除了某些信息未明确指定外,看起来就和普通函数一样。这些未指定的信息就是参数化信息。

\subsubsubsection{1.1.1\hspace{0.2cm}定义模板}

下面就是一个函数模板,返回两个数的最大值:

\noindent
\textit{basics/max1.hpp}
\begin{lstlisting}[style=styleCXX]
template<typename T>
T max (T a, T b)
{
	// if b < a then yield a else yield b
	return b < a ? a : b;
}
\end{lstlisting}

这个模板定义指定了一个函数组,它返回两个值的最大值,a和b作为函数的参数。

\begin{tcolorbox}[colback=webgreen!5!white,colframe=webgreen!75!black]
\hspace*{0.75cm}根据[StepanovNotes],\texttt{max()}模板有意地返回\texttt{b < a ? a: b},而不是\texttt{a < b ? b: a},是为了确保函数行为的正确性。
\end{tcolorbox}

形参的类型为\texttt{模板参数T}。本例中可以看到,模板参数必须使用以下形式的进行声明:

\begin{lstlisting}[style=styleCXX]
template<逗号分割的模板参数>
\end{lstlisting}

示例中,参数列表是\texttt{typename T}。注意\texttt{<}和\texttt{>}尖括号。关键字\texttt{typename}引入了一个类型参数。这是C++程序中最常见的模板参数类型,也可以使用其他参数,会在后面进行讨论(参见第3章)。

类型参数是\texttt{T},也可以使用其他标识符作为参数名,不过T是惯例罢了。类型参数表示在调用函数时才确定的类型,可以使用任意支持模板使用操作的类型(基本类型、类等)。因为a和b使用小于操作符进行比较,所以类型\texttt{T}必须支持操作符\texttt{<}。\texttt{max()}的定义可能不太明显,类型T的值必须可复制,以便返回。

\begin{tcolorbox}[colback=webgreen!5!white,colframe=webgreen!75!black]
\hspace*{0.75cm}C++17之前,类型\texttt{T}也必须可复制才能传递参数。C++17以后,即使复制构造函数和移动构造函数都无效,也可以传递临时变量(右值,参见附录B)。
\end{tcolorbox}

因为一些历史原因,还可以使用关键字\texttt{class}来定义类型参数。关键字\texttt{typename}在C++98标准中出现得相对较晚,所以在此之前,关键字\texttt{class}是引入类型参数的唯一方法,现在这种方法仍然有效。因此,模板\texttt{max()}可以等价地定义为:

\begin{lstlisting}[style=styleCXX]
template<class T>
T max (T a, T b)
{
	return b < a ? a : b;
}
\end{lstlisting}

这里没有语义上的区别,即使使用\texttt{class},模板参数也可以使用任意类型。但因为这样使用\texttt{class}可能会引起误解(不仅是类可以替换为\texttt{T}),所以应该更倾向于中使用\texttt{typename}。但与类声明不同,在声明类型参数时,不能使用关键字\texttt{struct}来代替\texttt{typename}。

\subsubsubsection{1.1.2\hspace{0.2cm}使用模板}

下面的展示了如何使用函数模板\texttt{max()}:

\noindent
\textit{basics/max1.cpp}
\begin{lstlisting}[style=styleCXX]
#include "max1.hpp"
#include <iostream>
#include <string>

int main()
{
	int i = 42;
	std::cout << "max(7,i): " << ::max(7,i) << '\n';
	
	double f1 = 3.4;
	double f2 = -6.7;
	std::cout << "max(f1,f2): " << ::max(f1,f2) << '\n';
	
	std::string s1 = "mathematics";
	std::string s2 = "math";
	std::cout << "max(s1,s2): " << ::max(s1,s2) << '\n';
}
\end{lstlisting}

这里,\texttt{max()}调用了三次:整型,双精度浮点类型,\texttt{std::string}。输出如下:

\begin{tcblisting}{commandshell={}}
max(7,i): 42
max(f1,f2): 3.4
max(s1,s2): mathematics
\end{tcblisting}

代码中,对\texttt{max()}的使用都用\texttt{::}限定。这是为了确保在全局命名空间中找到\texttt{max()}模板。在标准库中有一个\texttt{std::max()},在某些情况下可以使用,但在这里可能会产生歧义。

\begin{tcolorbox}[colback=webgreen!5!white,colframe=webgreen!75!black]
\hspace*{0.75cm}在命名空间\texttt{std}中定义了一个参数类型(例如\texttt{std::string}),根据C++的查找规则,\texttt{std}中的\texttt{max()}模板函数和全局模板函数都会被找到(参见附录C)。
\end{tcolorbox}

模板不会编译成处理任意类型的实体。相反,对于模板所使用的每个类型,会根据模板中生成不同的实体。

\begin{tcolorbox}[colback=webgreen!5!white,colframe=webgreen!75!black]
\hspace*{0.75cm}“一体多用”的方案想想是可以的,但不会在实践中使用(运行时效率较低)。所有语言规则都基于这样一个原则:对于不同的模板参数,会生成不同的实体。
\end{tcolorbox}

因此,\texttt{max()}对这三种类型进行了分别编译。例如,\texttt{max()}的第一次调用

\begin{lstlisting}[style=styleCXX]
int i = 42;
... max(7,i) ...
\end{lstlisting}

使用int作为模板形参T的函数模板。因此,等同于调用的如下函数:

\begin{lstlisting}[style=styleCXX]
int max (int a, int b)
{
	return b < a ? a : b;
}
\end{lstlisting}

用具体类型替换模板参数的过程称为实例化,会产生一个模板实例。

\begin{tcolorbox}[colback=webgreen!5!white,colframe=webgreen!75!black]
\hspace*{0.75cm}面向对象编程中,术语\texttt{实例(instance)}和\texttt{实例化(instantiate)}用于不同的上下文——类的具体对象。然而,由于本书针对模板,仅用来这两个术语表示对模板的“使用”,除非另有说明。
\end{tcolorbox}

注意,使用函数模板就可以进行实例化,开发者无需单独实例化。

类似地,\texttt{max()}的其他调用实例化了\texttt{double}和\texttt{std::string}的\texttt{max}模板:

\begin{lstlisting}[style=styleCXX]
double max (double, double);
std::string max (std::string, std::string);
\end{lstlisting}

如果结果代码有效,\texttt{void}是有效的模板参数。例如:

\begin{lstlisting}[style=styleCXX]
template<typename T>
T foo(T*)
{
}

void* vp = nullptr;
foo(vp); // OK: deduces void foo(void*)
\end{lstlisting}

\subsubsubsection{1.1.3\hspace{0.2cm}两阶段翻译}

An attempt to instantiate a template for a type that doesn’t support all the operations used within it will result in a compile-time error. For example:

\begin{lstlisting}[style=styleCXX]
std::complex<float> c1, c2; // doesn’t provide operator <
...
::max(c1,c2); // ERROR at compile time
\end{lstlisting}

Thus, templates are “compiled” in two phases:

\begin{enumerate}
\item 
Without instantiation at definition time, the template code itself is checked for correctness ignoring the template parameters. This includes:
\begin{itemize}
\item[-] 
Syntax errors are discovered, such as missing semicolons.

\item[-]
Using unknown names (type names, function names, ...) that don’t depend on template parameters are discovered.

\item[-]
Static assertions that don’t depend on template parameters are checked.
\end{itemize}

\item 
At instantiation time, the template code is checked (again) to ensure that all code is valid. That is, now especially, all parts that depend on template parameters are double-checked.
\end{enumerate}

For example:

\begin{lstlisting}[style=styleCXX]
template<typename T>
void foo(T t)
{
	undeclared(); // first-phase compile-time error if undeclared() unknown
	undeclared(t); // second-phase compile-time error if undeclared(T) unknown
	static_assert(sizeof(int) > 10, // always fails if sizeof(int)<=10
				  "int too small");
	static_assert(sizeof(T) > 10, // fails if instantiated for T with size <=10
	              "T too small");
}
\end{lstlisting}

The fact that names are checked twice is called two-phase lookup and discussed in detail in Section 14.3.1。

Note that some compilers don’t perform the full checks of the first phase.

\begin{tcolorbox}[colback=webgreen!5!white,colframe=webgreen!75!black]
\hspace*{0.75cm}For example, The Visual C++ compiler in some versions (such as Visual Studio 20133 and 2015) allow undeclared names that don’t depend on template parameters and even some syntax flaws (such as a missing semicolon).
\end{tcolorbox}

So you might not see general problems until the template code is instantiated at least once.

\hspace*{\fill} \\ %插入空行
\noindent
\textbf{Compiling and Linking}

Two-phase translation leads to an important problem in the handling of templates in practice: When a function template is used in a way that triggers its instantiation, a compiler will (at some point) need to see that template’s definition. This breaks the usual compile and link distinction for ordinary functions, when the declaration of a function is sufficient to compile its use. Methods of handling this problem are discussed in Chapter 9. For the moment, let’s take the simplest approach: Implement each template inside a header file.








