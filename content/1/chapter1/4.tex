还可以为模板参数定义默认值。这些值被称为默认模板参数,可以用于任何类型的模板。

\begin{tcolorbox}[colback=webgreen!5!white,colframe=webgreen!75!black]
\hspace*{0.75cm}在C++11之前,因为历史原因,默认模板参数只允许在类模板中使用。
\end{tcolorbox}

甚至可能引用前面的模板参数。

例如,如果希望定义返回类型的方法与具有多个参数类型的能力结合起来(如前一节所述),那么可以为返回类型引入一个模板参数RT,将两个参数的公共类型作为默认类型。同样,有多种选择:

\begin{enumerate}
\item 
可以直接使用三元操作符。但在使用三元操作符之前,只能使用参数的类型:

\noindent
\textit{basics/maxdefault1.hpp}
\begin{lstlisting}[style=styleCXX]
#include <type_traits>

template<typename T1, typename T2,
typename RT = std::decay_t<decltype(true ? T1() : T2())>>
RT max (T1 a, T2 b)
{
	return b < a ? a : b;
}
\end{lstlisting}

注意\texttt{std::decay\_t<>}的用法以确保返回的不是引用类型。

\begin{tcolorbox}[colback=webgreen!5!white,colframe=webgreen!75!black]
\hspace*{0.75cm}同样,在C++11中,必须使用typename \texttt{std::decay<…>::type},而非\texttt{std::decay\_t<…>}(见第2.8节)。
\end{tcolorbox}

此实现要求能够为传递的类型调用默认构造函数。还有另一种解决方案,使用\texttt{std::declval},但是这会使声明更加复杂。参见第11.2.3节中的示例。

\item
也可以使用\texttt{std::common\_type<>}来指定返回类型的默认值:

\noindent
\textit{basics/maxdefault3.hpp}
\begin{lstlisting}[style=styleCXX]
#include <type_traits>

template<typename T1, typename T2,
typename RT = std::common_type_t<T1,T2>>
RT max (T1 a, T2 b)
{
	return b < a ? a : b;
}
\end{lstlisting}

注意\texttt{std::common\_type<>}的衰减,因此返回值不会是引用。
\end{enumerate}

可以使用返回类型的默认值:

\begin{lstlisting}[style=styleCXX]
auto a = ::max(4, 7.2);
\end{lstlisting}

或者显式地在所有其他参数类型之后指定返回的类型:

\begin{lstlisting}[style=styleCXX]
auto b = ::max<double,int,long double>(7.2, 4);
\end{lstlisting}

不过,必须指定三种类型才能只指定返回类型。所以,我们需要将返回类型作为第一个模板参数,这样就可以对其类型进行推断。原则上,即使后面没有默认参数,导函数模板参数也可以有默认的类型:

\begin{lstlisting}[style=styleCXX]
template<typename RT = long, typename T1, typename T2>
RT max (T1 a, T2 b)
{
	return b < a ? a : b;
}
\end{lstlisting}

通过定义,你这样使用:

\begin{lstlisting}[style=styleCXX]
int i;
long l;
...
max(i, l); // 返回long(返回类型的模板默认类型)
max<int>(4, 42); // 显式返回int
\end{lstlisting}

但这种方法只有在模板参数有默认值的情况下才有意义。这里,需要模板参数的默认值依赖于之前的模板参数。原则上,这是可能的,在26.5.1节中会进行讨论,但是这种技术依赖于类型特征,并会使定义复杂化。

由于所有这些原因,最好和最简单的解决方案是让编译器推导出第1.3.2节中提出的返回类型。
