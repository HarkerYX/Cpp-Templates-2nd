C++17前,必须将所有模板参数类型传递给类模板(除非有默认值)。C++17后,必须明确指定模板参数的约束放宽了。相反,如果构造函数能够推导出所有模板形参(没有默认值),则可以不用显式定义模板参数。

例如,前面所有的代码示例中,可以使用复制构造函数,而非指定模板参数:

\begin{lstlisting}[style=styleCXX]
Stack<int> intStack1; // stack of strings
Stack<int> intStack2 = intStack1; // OK in all versions
Stack intStack3 = intStack1; // OK since C++17
\end{lstlisting}

通过接受参数的构造函数,支持推断堆栈的元素类型。例如,可以提供由单个元素初始化的堆栈:

\begin{lstlisting}[style=styleCXX]
template<typename T>
class Stack {
private:
	std::vector<T> elems; // elements
public:
	Stack () = default;
	Stack (T const& elem) // initialize stack with one element
	: elems({elem}) {
	}
	...
};
\end{lstlisting}

可以这样声明栈:

\begin{lstlisting}[style=styleCXX]
Stack intStack = 0; // Stack<int> deduced since C++17
\end{lstlisting}

使用整数0初始化堆栈,模板参数T推导为int,因此会实例化一个、\texttt{stack<int>}类型。

请注意以下几点:

\begin{itemize}
\item 
根据int构造函数的定义,必须请求默认构造函数以其默认行为可用,因为只有在没有定义其他构造函数时,默认构造函数才可用:
\begin{lstlisting}[style=styleCXX]
Stack() = default;
\end{lstlisting}

\item 
参数elem传递给带大括号的elems,以初始化初始化列表,以elem作为唯一参数:
\begin{lstlisting}[style=styleCXX]
: elems({elem})
\end{lstlisting}
\texttt{vector}的构造函数中无法接受单个参数作为初始元素。

\begin{tcolorbox}[colback=webgreen!5!white,colframe=webgreen!75!black]
\hspace*{0.75cm}更糟糕的是,\texttt{vector}的构造函数会将一个整型参数作为初始大小,因此对于初始值为5的堆栈,当使用\texttt{:elems(elem)}时,\texttt{vector}将获得5个元素的初始大小。
\end{tcolorbox}

\end{itemize}

与函数模板不同,类模板参数不能只推导部分参数(通过显式地只指定一些模板参数)。详细信息请参见15.12。

\hspace*{\fill} \\ %插入空行
\noindent
\textbf{用字符串字面值推导类模板参数}

原则上,可以用字符串字面值初始化堆栈:

\begin{lstlisting}[style=styleCXX]
Stack stringStack = "bottom"; // Stack<char const[7]> deduced since C++17
\end{lstlisting}

但这会带来很多麻烦:通常,通过引用传递模板类型T的参数时,参数不会衰变,衰变是指将原始数组类型转换为相应的原始指针类型的机制。这意味着会初始化了一个

\begin{lstlisting}[style=styleCXX]
Stack<char const[7]>
\end{lstlisting}

这是应为,在使用T的地方使用了\texttt{char const[7]}。例如,我们不能推入不同大小的字符串,因为它有不同的类型。有关详细讨论,请参阅第7.4节。

但当按值传递模板类型T的阐述时,参数会衰变,将原始数组类型转换为相应的原始指针类型。也就是说,构造函数的调用参数T推导为\texttt{char const*},因此整个类将推导为\texttt{Stack<char const*>}。

由于这个原因,可以声明构造函数,直接传递值:

\begin{lstlisting}[style=styleCXX]
class Stack {
private:
	std::vector<T> elems; // elements
public:
	Stack (T elem) // initialize stack with one element by value
	: elems({elem}) { // to decay on class tmpl arg deduction
	}
	...
};
\end{lstlisting}

有了这个,下面的初始化就可以工作了:

\begin{lstlisting}[style=styleCXX]
Stack stringStack = "bottom"; // Stack<char const*> deduced since C++17
\end{lstlisting}

然而,在这种情况下,最好将临时元素移到堆栈中,以避免不必要地复制:

\begin{lstlisting}[style=styleCXX]
template<typename T>
class Stack {
private:
	std::vector<T> elems; // elements
public:
	Stack (T elem) // initialize stack with one element by value
	: elems({std::move(elem)}) {
	}
	...
};
\end{lstlisting}

\hspace*{\fill} \\ %插入空行
\noindent
\textbf{推导策略}

除了声明构造函数由值调用之外,还有一种不同的解决方案:因为在容器中处理指针是噩梦的开始,所以应该禁止为容器类自动推导字符指针。

可以定义特定的推导策略来提供额外的,或修复现有类模板参数的推导。例如,可以定义只要传入一个字符串字面值或C字符串,栈就会实例化为\texttt{std::string}:

\begin{lstlisting}[style=styleCXX]
Stack(char const*) -> Stack<std::string>;
\end{lstlisting}

这个策略必须出现在与类定义相同的作用域(命名空间)中。通常,它遵循类定义。将\texttt{->}后面的类型称为推导策略的引导类型。

现在,声明

\begin{lstlisting}[style=styleCXX]
Stack stringStack{"bottom"}; // OK: Stack<std::string> deduced since C++17
\end{lstlisting}

将堆栈推断为一个\texttt{Stack<std::string>}类。然而,在以下情况下仍然不能正常工作:

\begin{lstlisting}[style=styleCXX]
Stack stringStack = "bottom"; // Stack<std::string> deduced, but still not valid
\end{lstlisting}

推导\texttt{std::string}以便实例化一个\texttt{Stack<std::string>}类:

\begin{lstlisting}[style=styleCXX]
class Stack {
private:
	std::vector<std::string> elems; // elements
public:
	Stack (std::string const& elem) // initialize stack with one element
	: elems({elem}) {
	}
	...
};
\end{lstlisting}

但根据语言规则,不能通过将字符串字面值传递给\texttt{std::string}的构造函数来复制对象,从而完成初始化(initialize using =)。所以必须像这样初始化堆栈:

\begin{lstlisting}[style=styleCXX]
Stack stringStack{"bottom"}; // Stack<std::string> deduced and valid
\end{lstlisting}

此时的疑问应该在于,类模板参数推导的副本。在将\texttt{stringStack}声明为\texttt{Stack<std::string>}之后,下面的初始化声明了相同的类型(因此,调用了复制构造函数),而不是通过字符串堆栈元素对堆栈进行初始化:

\begin{lstlisting}[style=styleCXX]
Stack stack2{stringStack}; // Stack<std::string> deduced
Stack stack3(stringStack); // Stack<std::string> deduced
Stack stack4 = {stringStack}; // Stack<std::string> deduced
\end{lstlisting}

有关类模板参数推导的更多细节,请参见15.12节。





















