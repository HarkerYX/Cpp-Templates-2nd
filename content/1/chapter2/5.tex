可以为某些模板参数特化类模板。类似于函数模板的重载(参见第1.5节),特化类模板允许优化特定类型的实现,或者为类模板的实例化修复特定类型的错误行为。但若特化类模板,则必须特化所有成员函数。虽然可以特化类模板的单个成员函数,但若这样做了,就不能再特化该特化成员所属的整个类模板实例。

要特化类模板,必须用template<>和类模板特化的类型声明类。这些类型可用作模板参数,必须直接在类名之后指定:

\begin{lstlisting}[style=styleCXX]
template<>
class Stack<std::string> {
	...
};
\end{lstlisting}

对于这些特化,成员函数的定义必须是一个“普通的”成员函数,每次出现T都会使用特化类型替换:

\begin{lstlisting}[style=styleCXX]
void Stack<std::string>::push (std::string const& elem)
{
	elems.push_back(elem); // append copy of passed elem
}
\end{lstlisting}

下面是std::string类型的Stack<>特化:

\hspace*{\fill} \\ %插入空行
\noindent
\textit{basics/stack2.hpp}
\begin{lstlisting}[style=styleCXX]
#include "stack1.hpp"
#include <deque>
#include <string>
#include <cassert>

template<>
class Stack<std::string> {
	private:
	std::deque<std::string> elems; // elements
	public:
	void push(std::string const&); // push element
	void pop(); // pop element
	std::string const& top() const; // return top element
	bool empty() const { // return whether the stack is empty
		return elems.empty();
	}
};

void Stack<std::string>::push (std::string const& elem)
{
	elems.push_back(elem); // append copy of passed elem
}

void Stack<std::string>::pop ()
{
	assert(!elems.empty());
	elems.pop_back(); // remove last element
}

std::string const& Stack<std::string>::top () const
{
	assert(!elems.empty());
	return elems.back(); // return copy of last element
}
\end{lstlisting}

例子中,特化使用引用语义将字符串参数传递给push(),这对这个特定类型有意义(不过,应该更好地传递转发引用,将在第6.1节中讨论)。

另一个区别是使用deque(而非vector)来管理堆栈内的元素。尽管这在这里没有什么区别,但说明了特化实现可能与主模板实现完全不同。





















