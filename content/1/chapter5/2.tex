For fundamental types such as int, double, or pointer types, there is no default constructor that initializes them with a useful default value. Instead, any noninitialized local variable has an undefined value:

\begin{lstlisting}[style=styleCXX]
void foo()
{
	int x; // x has undefined value
	int* ptr; // ptr points to anywhere (instead of nowhere)
}
\end{lstlisting}

Now if you write templates and want to have variables of a template type initialized by a default value, you have the problem that a simple definition doesn’t do this for built-in types:

\begin{lstlisting}[style=styleCXX]
template<typename T>
void foo()
{
	T x; // x has undefined value if T is built-in type
}
\end{lstlisting}

For this reason, it is possible to call explicitly a default constructor for built-in types that initializes them with zero (or false for bool or nullptr for pointers). As a consequence, you can ensure proper initialization even for built-in types by writing the following:

\begin{lstlisting}[style=styleCXX]
template<typename T>
void foo()
{
	T x{}; // x is zero (or false) if T is a built-in type
}
\end{lstlisting}

This way of initialization is called value initialization, which means to either call a provided constructor or zero initialize an object. This even works if the constructor is explicit.

Before C++11, the syntax to ensure proper initialization was

\begin{lstlisting}[style=styleCXX]
T x = T(); // x is zero (or false) if T is a built-in type
\end{lstlisting}

Prior to C++17, this mechanism (which is still supported) only worked if the constructor selected for the copy-initialization is not explicit. In C++17, mandatory copy elision avoids that limitation and either syntax can work, but the braced initialized notation can use an initializer-list constructor if no default constructor is available.

\begin{tcolorbox}[colback=webgreen!5!white,colframe=webgreen!75!black]
\hspace*{0.75cm}That is, a constructor with a parameter of type std::initializer\_list<X>, for some type X.
\end{tcolorbox}

To ensure that a member of a class template, for which the type is parameterized, gets initialized, you can define a default constructor that uses a braced initializer to initialize the member:

\begin{lstlisting}[style=styleCXX]
template<typename T>
class MyClass {
private:
	T x;
public:
	MyClass() : x{} { // ensures that x is initialized even for built-in types
	}
	...
};
\end{lstlisting}

The pre-C++11 syntax

\begin{lstlisting}[style=styleCXX]
MyClass() : x() { // ensures that x is initialized even for built-in types
}
\end{lstlisting}

also still works.

Since C++11, you can also provide a default initialization for a nonstatic member, so that the following is also possible:

\begin{lstlisting}[style=styleCXX]
template<typename T>
class MyClass {
private:
	T x{}; // zero-initialize x unless otherwise specified
	...
};
\end{lstlisting}

However, note that default arguments cannot use that syntax. For example,

\begin{lstlisting}[style=styleCXX]
template<typename T>
void foo(T p{}) { // ERROR
	...
}
\end{lstlisting}

Instead, we have to write:

\begin{lstlisting}[style=styleCXX]
template<typename T>
void foo(T p = T{}) { // OK (must use T() before C++11)
	...
}
\end{lstlisting}





















