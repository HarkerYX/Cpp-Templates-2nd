包扩展(在第12.4.1节中有详细描述)是一种机制,可将类型列表迭代的工作转移给编译器。因为需要对列表中的每个元素应用相同的操作(进行包扩展),这里可以使用在第24.2.5节中开发的Transform算法。这为类型列表的Transform提供了一个特化算法(通过偏特化):

\hspace*{\fill} \\ %插入空行
\noindent
\textit{typelist/variadictransform.hpp}
\begin{lstlisting}[style=styleCXX]
template<typename... Elements, template<typename T> class MetaFun>
class TransformT<Typelist<Elements...>, MetaFun, false>
{
	public:
	using Type = Typelist<typename MetaFun<Elements>::Type...>;
};
\end{lstlisting}

这个实现将类型列表元素捕获到一个参数包elements中,使用模式typename MetaFun<Elements>::Type进行包扩展,将元功能应用到Elements中的每个类型,并形成一个类型列表。因为它不需要递归,所以这个实现更简单,并且以一种相当直接的方式使用语言特性。因为只需要实例化Transform模板的一个实例,所以需要更少的模板实例化。该算法仍然需要线性数量的MetaFun实例化,这些实例化是算法的基础。

其他算法间接受益于使用包扩展,在第24.2.4节中描述的反向算法需要线性数量的PushBack实例化。在那一节中描述的类型列表PushBack的包扩展形式(需要单个实例化),Reverse是线性的。然而,描述的Reverse更一般的递归实现本身在实例化数量上是线性的,使Reverse的复杂度与数量成平方关系(2次方)!

选择给定索引列表中的元素以生成新的类型列表时,可以使用Pack扩展。Select元函数接受一个类型列表和一个包含该类型列表索引的值列表,然后生成一个包含由值列表指定元素的新类型列表:

\hspace*{\fill} \\ %插入空行
\noindent
\textit{typelist/select.hpp}
\begin{lstlisting}[style=styleCXX]
template<typename Types, typename Indices>
class SelectT;

template<typename Types, unsigned... Indices>
class SelectT<Types, Valuelist<unsigned, Indices...>>
{
	public:
	using Type = Typelist<NthElement<Types, Indices>...>;
};

template<typename Types, typename Indices>
using Select = typename SelectT<Types, Indices>::Type;
\end{lstlisting}

索引在参数包Indices中捕获,该参数包扩展以生成索引到给定类型列表的一系列NthElement类型,并在新参数列表中捕获结果。下面的例子说明了如何使用Select来反转输入列表:

\begin{lstlisting}[style=styleCXX]
using SignedIntegralTypes =
	Typelist<signed char, short, int, long, long long>;

using ReversedSignedIntegralTypes =
	Select<SignedIntegralTypes, Valuelist<unsigned, 4, 3, 2, 1, 0>>;
	// produces Typelist<long long, long, int, short, signed char>
\end{lstlisting}

包含另一个列表索引的非类型类型列表,称为索引列表(或索引序列),允许简化或消除递归计算。索引列表在25.3.4节有详细的描述。










































