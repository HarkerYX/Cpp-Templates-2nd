
早期的C++,枚举值是在类声明中创建“真正的常量”(称为常量表达式)作为命名成员的唯一机制。有了它们,可以定义一个Pow3元程序来计算3的幂,如下所示:

\hspace*{\fill} \\ %插入空行
\noindent
\textit{meta/pow3enum.hpp}
\begin{lstlisting}[style=styleCXX]
// primary template to compute 3 to the Nth
template<int N>
struct Pow3 {
	enum { value = 3 * Pow3<N-1>::value };
};

// full specialization to end the recursion
template<>
struct Pow3<0> {
	enum { value = 1 };
};
\end{lstlisting}

C++98的标准化引入了类内静态常量初始化器的概念,因此Pow3元程序可以这样写:

\hspace*{\fill} \\ %插入空行
\noindent
\textit{meta/pow3const.hpp}
\begin{lstlisting}[style=styleCXX]
// primary template to compute 3 to the Nth
template<int N>
struct Pow3 {
	static int const value = 3 * Pow3<N-1>::value;
};

// full specialization to end the recursion
template<>
struct Pow3<0> {
	static int const value = 1;
};
\end{lstlisting}

但是,这个版本有一个缺点:静态常量成员是左值(参见附录B)

\begin{lstlisting}[style=styleCXX]
void foo(int const&);
\end{lstlisting}

将元程序的结果传递给它:

\begin{lstlisting}[style=styleCXX]
foo(Pow3<7>::value);
\end{lstlisting}

编译器必须传递Pow3<7>::value的地址,这迫使编译器实例化和分配静态成员的定义。因此,计算不再局限于纯粹的“编译时”效应。

枚举值不是左值(即,它们没有地址)。因此,当通过引用传递它们时,没有使用静态内存。这就像将计算值作为文字传递一样。因此,本书的第一版在这类应用程序中更倾向于使用枚举常量。

然而,C++11引入了constexpr静态数据成员,而且这些成员不限于整型。它们没有解决上面提出的地址问题,尽管有这个缺点,不过目前是生成元程序结果的常用方法。它们的优点是具有正确的类型(与人工enum类型相反),并且当使用auto类型说明符声明静态成员时,可以推导出该类型。C++17添加了内联静态数据成员,这确实解决了上面提出的地址问题,并且可以与constexpr一起使用。























