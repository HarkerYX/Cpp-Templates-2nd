Metaprogramming consists of “programming a program.” In other words, we lay out code that the programming system executes to generate new code that implements the functionality we really want. Usually, the term metaprogramming implies a reflexive attribute: The metaprogramming component is part of the program for which it generates a bit of code (i.e., an additional or different bit of the program).

Why would metaprogramming be desirable? As with most other programming techniques, the goal is to achieve more functionality with less effort, where effort can be measured as code size, maintenance cost, and so forth. What characterizes metaprogramming is that some user-defined computation happens at translation time. The underlying motivation is often performance (things computed at translation time can frequently be optimized away) or interface simplicity (a metaprogram is generally shorter than what it expands to) or both.

Metaprogramming often relies on the concepts of traits and type functions, as developed in Chapter 19. We therefore recommend becoming familiar with that chapter prior to delving into this one.