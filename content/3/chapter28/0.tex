Templates raise two classes of challenges when it comes to debugging them. One set of challenges is definitely a problem for writers of templates: How can we ensure that the templates we write will function for any template arguments that satisfy the conditions we document? The other class of problems is almost exactly the opposite: How can a user of a template find out which of the template parameter requirements it violated when the template does not behave as documented?

Before we discuss these issues in depth, it is useful to contemplate the kinds of constraints that may be imposed on template parameters. In this chapter, we deal mostly with the constraints that lead to compilation errors when violated, and we call these constraints syntactic constraints. Syntactic constraints can include the need for a certain kind of constructor to exist, for a particular function call to be unambiguous, and so forth. The other kind of constraint we call semantic constraints. These constraints are much harder to verify mechanically. In the general case, it may not even be practical to do so. For example, we may require that there be a < operator defined on a template type parameter (which is a syntactic constraint), but usually we’ll also require that the operator actually defines some sort of ordering on its domain (which is a semantic constraint).

The term concept is often used to denote a set of constraints that is repeatedly required in a template library. For example, the C++ standard library relies on such concepts as random access iterator and default constructible. With this terminology in place, we can say that debugging template code includes a significant amount of determining how concepts are violated in the template implementation and in their use. This chapter delves into both design and debugging techniques that can make templates easier to work with, both for template authors and for template users.