当模板错误发生时,问题通常在实例化后发现,从而会有冗长的错误消息,就像在第9.4节中那样。

\begin{tcolorbox}[colback=webgreen!5!white,colframe=webgreen!75!black]
\hspace*{0.75cm}到这里,无疑在写代码时会遇到一些错误消息,会使最初的示例看起来很乏味!
\end{tcolorbox}

为了说明这一点,请考虑以下的手写代码:

\begin{lstlisting}[style=styleCXX]
template<typename T>
void clear (T& p)
{
	*p = 0; // assumes T is a pointer-like type
}

template<typename T>
void core (T& p)
{
	clear(p);
}

template<typename T>
void middle (typename T::Index p)
{
	core(p);
}

template<typename T>
void shell (T const& env)
{
	typename T::Index i;
	middle<T>(i);
}
\end{lstlisting}

这个例子阐明了软件开发的分层:像shell()这样的高级函数模板依赖于像middle()这样的组件,而这些组件本身会使用像core()这样的功能。当实例化shell()时,下面的所有层也需要实例化。这个例子中,有一个问题:core()实例化为int类型(在middle()中使用Client::Index),并试图错误的解引用该类型。

该错误仅在实例化时可检测到。例如:

\begin{lstlisting}[style=styleCXX]
class Client
{
	public:
	using Index = int;
};

int main()
{
	Client mainClient;
	shell(mainClient);
}
\end{lstlisting}

好的通用诊断包括导致问题的所有级别的跟踪,但是获得这么多信息,也会让我们感觉手足无措。

在[StroustrupDnE]中可以找到围绕这个问题核心思想的讨论,Bjarne Stroustrup确定了两类方法来更早地确定模板参数是否满足一组约束:通过语言扩展或更早的参数使用。我们在第17.8节和附录E中介绍了前一种选择,后一种选择包括在浅层实例化中强制错误。这通过插入未使用的代码来实现,如果代码使用的模板参数不满足更深层模板的要求,那么就会触发错误。

前面的例子中,可以在shell()中添加代码,尝试对T::Index类型的值解引用。例如:

\begin{lstlisting}[style=styleCXX]
template<typename T>
void ignore(T const&)
{ }

template<typename T>
void shell (T const& env)
{
	class ShallowChecks
	{
		void deref(typename T::Index ptr) {
			ignore(*ptr);
		}
	};
	typename T::Index i;
	middle(i);
}
\end{lstlisting}

若T是不能解引用T::Index的类型,则会在局部类ShallowChecks上出现编译错误。因为没有使用局部类,所以添加的代码不会影响shell()函数的运行时间,但许多编译器会警告说没有使用ShallowChecks(成员也是如此)。可以使用ignore()模板等技巧来抑制此类警告,但也会增加代码的复杂性。

\hspace*{\fill} \\ %插入空行
\noindent
\textbf{概念检查}

显然,示例中的代码开发可能会变得与实现模板的实际功能代码一样复杂。为了控制这种复杂性,可以尝试在某种类型的库中收集各种代码片段。例如,这样的库可以包含宏,当模板参数替换违反该特定参数的概念时,这些宏可以扩展为触发适当错误的代码。这类库中最流行的是Concept Check库,是Boost发行版的一部分(参见[BCCL])。但这种技术的可移植性不是特别好(不同编译器诊断错误的方式不同),有时还会掩盖在更高级别上无法捕获错误的问题。

当C++中有了概念(参见附录E),就有了其他的方法来支持需求和预期行为的定义。






























