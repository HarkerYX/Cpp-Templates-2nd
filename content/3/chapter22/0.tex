Chapter 18 described the nature of static polymorphism (via templates) and dynamic polymorphism (via inheritance and virtual functions) in C++. Both kinds of polymorphism provide powerful abstractions for writing programs, yet each has tradeoffs: Static polymorphism provides the same performance as nonpolymorphic code, but the set of types that can be used at run time is fixed at compile time. On the other hand, dynamic polymorphism via inheritance allows a single version of the polymorphic function to work with types not known at the time it is compiled, but it is less flexible because types must inherit from the common base class.

This chapter describes how to bridge between static and dynamic polymorphism in C++, providing some of the benefits discussed in Section 18.3 on page 375 from each model: the smaller executable code size and (almost) entirely compiled nature of dynamic polymorphism, along with the interface flexibility of static polymorphism that allows, for example, built-in types to work seamlessly. As an example, we will build a simplified version of the standard library’s function<> template.