
既然有了变量存储问题的解决方案,就可以设计Variant类型了。与Tuple类型一样,可以使用继承来为Types列表提供每个类型的行为。然而,与Tuple不同的是,这些基类不会存储。相反,每个基类都使用21.2节中讨论的奇异递归模板模式(CRTP),通过派生最多的类型访问共享变量。

下面定义的类模板VariantChoice提供了在变量的活动值为(或将为)T类型时,对缓冲区进行操作所需的操作:

\hspace*{\fill} \\ %插入空行
\noindent
\textit{variant/variantchoice.hpp}
\begin{lstlisting}[style=styleCXX]
#include "findindexof.hpp"
template<typename T, typename... Types>
class VariantChoice {
	using Derived = Variant<Types...>;
	Derived& getDerived() { return *static_cast<Derived*>(this); }
	Derived const& getDerived() const {
		return *static_cast<Derived const*>(this);
	}
	protected:
	// compute the discriminator to be used for this type
	constexpr static unsigned Discriminator =
		FindIndexOfT<Typelist<Types...>, T>::value + 1;
	public:
	VariantChoice() { }
	VariantChoice(T const& value); // see variantchoiceinit.hpp
	VariantChoice(T&& value); // see variantchoiceinit.hpp
	bool destroy(); // see variantchoicedestroy.hpp
	Derived& operator= (T const& value); // see variantchoiceassign.hpp
	Derived& operator= (T&& value); // see variantchoiceassign.hpp
};
\end{lstlisting}

模板参数包Types将包含变体中的所有类型,其可以形成Derived类型(用于CRTP),从而提供向下转换操作getDerived()。类型的第二个有趣用法是,在类型列表中查找特定类型T的位置,这可以通过元函数FindIndexOfT完成:

\hspace*{\fill} \\ %插入空行
\noindent
\textit{variant/findindexof.hpp}
\begin{lstlisting}[style=styleCXX]
template<typename List, typename T, unsigned N = 0,
		bool Empty = IsEmpty<List>::value>
struct FindIndexOfT;

// recursive case:
template<typename List, typename T, unsigned N>
struct FindIndexOfT<List, T, N, false>
: public IfThenElse<std::is_same<Front<List>, T>::value,
					std::integral_constant<unsigned, N>,
					FindIndexOfT<PopFront<List>, T, N+1>>
{
};

// basis case:
template<typename List, typename T, unsigned N>
struct FindIndexOfT<List, T, N, true>
{
};
\end{lstlisting}

该指标值用于计算T对应的辨别器的值;稍后我们将返回特定的判别值。

Variant的框架如下,表明了Variant、VariantStorage和VariantChoice之间的关系:

\hspace*{\fill} \\ %插入空行
\noindent
\textit{variant/variant-skel.hpp}
\begin{lstlisting}[style=styleCXX]
template<typename... Types>
class Variant
	: private VariantStorage<Types...>,
	private VariantChoice<Types, Types...>...
{
	template<typename T, typename... OtherTypes>
		friend class VariantChoice; // enable CRTP
	...
};
\end{lstlisting}

如前所述,每个Variant都有一个单独的、共享的VariantStorage基类。

\begin{tcolorbox}[colback=webgreen!5!white,colframe=webgreen!75!black]
\hspace*{0.75cm}基类是私有的,因为它们不是公共接口的一部分。友元模板需要VariantChoice中的asDerived()函数,才能向下转换到Variant。
\end{tcolorbox}

此外,还有一些VariantChoice基类,是由以下嵌套的包扩展产生的(参见第12.4.4节):

\begin{lstlisting}[style=styleCXX]
VariantChoice<Types, Types...>...
\end{lstlisting}

这个实例中,有两个展开:外层展开,通过展开对类型的第一个引用,为类型中的每个类型T生成VariantChoice基类;内部展开,展开了第二次出现的Types,另外还将Types中的所有类型传递给每个VariantChoice基类。对于

\begin{lstlisting}[style=styleCXX]
Variant<int, double, std::string>
\end{lstlisting}

会产生下面一组VariantChoice基类:

\begin{tcolorbox}[colback=webgreen!5!white,colframe=webgreen!75!black]
\hspace*{0.75cm}通过类型T来区分给定Variant的VariantChoice基类的效果是,可以防止类型重复。Variant<double, int, double>将产生编译时错误,指出类不能直接继承同一基类(在本例中,VariantChoice<double, double, int, double>会出现两个编译错误)。
\end{tcolorbox}

\begin{lstlisting}[style=styleCXX]
VariantChoice<int, int, double, std::string>,
VariantChoice<double, int, double, std::string>,
VariantChoice<std::string, int, double, std::string>
\end{lstlisting}

这三个基类的辨别器值将分别为1、2和3。当变量存储的discriminator成员与特定VariantChoice基类的discriminator匹配时,该基类负责管理动态值。

标识值0是为Variant不包含值的情况保留,这是一种奇怪的状态,只有在赋值期间,抛出异常时才能观察到。对Variant的讨论中,将小心处理为0的辨别值(并在适当的时候进行设置),我们将这种情况的讨论留到第26.4.3节。

Variant的完整定义如下所示。下面几节将描述Variant每个成员的实现。

\hspace*{\fill} \\ %插入空行
\noindent
\textit{variant/variant.hpp}
\begin{lstlisting}[style=styleCXX]
template<typename... Types>
class Variant
: private VariantStorage<Types...>,
  private VariantChoice<Types, Types...>...
{
	template<typename T, typename... OtherTypes>
	friend class VariantChoice;
	
	public:
	template<typename T> bool is() const; // see variantis.hpp
	template<typename T> T& get() &; // see variantget.hpp
	template<typename T> T const& get() const&; // see variantget.hpp
	template<typename T> T&& get() &&; // see variantget.hpp
	
	// see variantvisit.hpp:
	template<typename R = ComputedResultType, typename Visitor>
		VisitResult<R, Visitor, Types&...> visit(Visitor&& vis) &;
	template<typename R = ComputedResultType, typename Visitor>
		VisitResult<R, Visitor, Types const&...> visit(Visitor&& vis) const&;
	template<typename R = ComputedResultType, typename Visitor>
		VisitResult<R, Visitor, Types&&...> visit(Visitor&& vis) &&;
	
	using VariantChoice<Types, Types...>::VariantChoice...;
	Variant(); // see variantdefaultctor.hpp
	Variant(Variant const& source); // see variantcopyctor.hpp
	Variant(Variant&& source); // see variantmovector.hpp
	template<typename... SourceTypes>
		Variant(Variant<SourceTypes...> const& source); // variantcopyctortmpl.hpp
	template<typename... SourceTypes>
		Variant(Variant<SourceTypes...>&& source);
	
	using VariantChoice<Types, Types...>::operator=...;
	Variant& operator= (Variant const& source); // see variantcopyassign.hpp
	Variant& operator= (Variant&& source);
	template<typename... SourceTypes>
		Variant& operator= (Variant<SourceTypes...> const& source);
	template<typename... SourceTypes>
		Variant& operator= (Variant<SourceTypes...>&& source);
	
	bool empty() const;
	
	~Variant() { destroy(); }
	void destroy(); // see variantdestroy.hpp
};
\end{lstlisting}








