Bill Gibbons was the main sponsor behind the introduction of the EBCO into the C++ programming language. Nathan Myers made it popular and proposed a template similar to our BaseMemberPair to take better advantage of it. The Boost library contains a considerably more sophisticated template, called compressed\_pair, that resolves some of the problems we reported for the MyClass template in this chapter. boost::compressed\_pair can also be used instead of our BaseMemberPair.

CRTP has been in use since at least 1991. However, James Coplien was first to describe them formally as a class of patterns (see [CoplienCRTP]). Since then, many applications of CRTP have been published. The phrase parameterized inheritance is sometimes wrongly equated with CRTP. As we have shown, CRTP does not require the derivation to be parameterized at all, and many forms of parameterized inheritance do not conform to CRTP. CRTP is also sometimes confused with the Barton-Nackman trick (see Section 21.2.1 on page 497) because Barton and Nackman frequently used CRTP in combination with friend name injection (and the latter is an important component of the Barton-Nackman trick). Our use of CRTP with the Barton-Nackman trick to provide operator implementations follows the same basic approach as the Boost.Operators library ([BoostOperators]), which provides an extensive set of operator definitions. Similarly, our treatment of iterator facades follows that of the Boost.Iterator library ([BoostIterator]) which provides a rich, standard-library compliant iterator interface for a derived type that provides a few core iterator operations (equality, dereference, movement), and also addresses tricky issues involving proxy iterators (which we did not address for the sake of brevity). Our ObjectCounter example is almost identical to a technique developed by Scott Meyers in [MeyersCounting].

The notion of mixins has been around in Object-Oriented programming since at least 1986 ([MoonFlavors]) as a way to introduce small pieces of functionality into an OO class. The use of templates for mixins in C++ became popular shortly after the first C++ standard was published, with two papers ([SmaragdakisBatoryMixins] and [EiseneckerBlinnCzarnecki]) describing the approaches commonly used today for mixins. Since then, it’s become a popular technique in C++ library design.

Named template arguments are used to simplify certain class templates in the Boost library. Boost uses metaprogramming to create a type with properties similar to our PolicySelector (but without using virtual inheritance). The simpler alternative presented here was developed by one of us (Vandevoorde).