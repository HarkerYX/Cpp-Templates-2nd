
简单的Polygon类,由一系列Point组成:

\begin{lstlisting}[style=styleCXX]
class Point
{
	public:
	double x, y;
	Point() : x(0.0), y(0.0) { }
	Point(double x, double y) : x(x), y(y) { }
};

class Polygon
{
	private:
	std::vector<Point> points;
	public:
	... // public operations
};
\end{lstlisting}

若用户可以扩展与每个Point关联的信息,包括应用程序特定的数据,比如:每个Point的颜色,或者可能将标签与每个Point相关联。可以根据Point的类型参数化Polygon,使这个扩展成为可能:

\begin{lstlisting}[style=styleCXX]
template<typename P>
class Polygon
{
	private:
	std::vector<P> points;
	public:
	... // public operations
};
\end{lstlisting}

用户可以使用继承创建自己的Point数据类型,提供与Point相同的接口,但包含其他应用程序特定的数据:

\begin{lstlisting}[style=styleCXX]
class LabeledPoint : public Point
{
	public:
	std::string label;
	LabeledPoint() : Point(), label("") { }
	LabeledPoint(double x, double y) : Point(x, y), label("") { }
};
\end{lstlisting}

这种实现有其缺点。首先,要求向用户公开Point类型,以便用户可以从它派生。此外,LabeledPoint的作者需要小心地提供与Point完全相同的接口(例如,继承或提供与Point相同的所有构造函数),否则LabeledPoint将无法与Polygon一起工作。若Point从一个版本的Polygon模板更改为另一个,这个约束就会有问题:添加新的Point构造函数可能需要更新每个派生类。

混合类提供了另一种方法来定制类型的行为,而不继承。混合类本质上颠倒了继承的方向,因为新类作为类模板的基类“混合”到继承层次中,而不是作为新的派生类创建。这种方法允许引入新的数据成员和其他操作,而不需要复制任何接口。

支持混合的类模板通常会接受任意数量的类,并从中派生:

\begin{lstlisting}[style=styleCXX]
template<typename... Mixins>
class Point : public Mixins...
{
	public:
	double x, y;
	Point() : Mixins()..., x(0.0), y(0.0) { }
	Point(double x, double y) : Mixins()..., x(x), y(y) { }
};
\end{lstlisting}

现在,可以“混合”一个包含标签的基类来生成一个LabeledPoint:

\begin{lstlisting}[style=styleCXX]
class Label
{
	public:
	std::string label;
	Label() : label("") { }
};

using LabeledPoint = Point<Label>;
\end{lstlisting}

或者混合几个基类:

\begin{lstlisting}[style=styleCXX]
class Color
{
	public:
	unsigned char red = 0, green = 0, blue = 0;
};

using MyPoint = Point<Label, Color>;
\end{lstlisting}

有了这个基于混合的Point,在不改变接口的情况下,可以很容易地向Point引入其他信息,所以Polygon很容易使用和扩展。用户只需要将特化的Point隐式转换到自定义混合类(上面的Label或Color),就可以访问该数据或接口。此外,Point类可以隐藏,并混合提供给Polygon类模板本身:

\begin{lstlisting}[style=styleCXX]
template<typename... Mixins>
class Polygon
{
	private:
	std::vector<Point<Mixins...>> points;
	public:
	... // public operations
};
\end{lstlisting}

模板需要进行少量定制的情况下,混合类很有用——比如适用用户指定的数据存储内部对象——而不需要库公开和记录这些内部数据类型及其接口。

\subsubsubsection{21.3.1\hspace{0.2cm}奇怪的混合类}

通过将混合类与第21.2节描述CRTP相结合,混合类可以更强大。每个混合实际上都是一个类模板,将与派生类的类型一起提供,从而可以对派生类进行定制。一个CRTP-混合版本的Point可以这样:

\begin{lstlisting}[style=styleCXX]
template<template<typename>... Mixins>
class Point : public Mixins<Point>...
{
	public:
	double x, y;
	Point() : Mixins<Point>()..., x(0.0), y(0.0) { }
	Point(double x, double y) : Mixins<Point>()..., x(x), y(y) { }
};
\end{lstlisting}

公式需要对要混合的每个类做更多工作,因此像Label和Color这样的类要成为类模板。但混合类可以根据混合到的派生类特定实例,调整其行为。可以将前面讨论的ObjectCounter模板混合到Point中,以计算Polygon创建的点的数量,并将该混合类与其他类混合在一起。

\subsubsubsection{21.3.2\hspace{0.2cm}参数化的虚拟性}

混合类还允许间接参数化派生类的其他属性,例如成员函数的虚函数。一个简单的例子展示了这种奇特的技巧:

\hspace*{\fill} \\ %插入空行
\noindent
\textit{inherit/virtual.cpp}
\begin{lstlisting}[style=styleCXX]
#include <iostream>

class NotVirtual {
};

class Virtual {
	public:
	virtual void foo() {
	}
};

template<typename... Mixins>
class Base : public Mixins... {
	public:
	// the virtuality of foo() depends on its declaration
	// (if any) in the base classes Mixins...
	void foo() {
		std::cout << "Base::foo()" << ’\n’;
	}
};

template<typename... Mixins>
class Derived : public Base<Mixins...> {
	public:
	void foo() {
		std::cout << "Derived::foo()" << ’\n’;
	}
};

int main()
{
	Base<NotVirtual>* p1 = new Derived<NotVirtual>;
	p1->foo(); // calls Base::foo()
	
	Base<Virtual>* p2 = new Derived<Virtual>;
	p2->foo(); // calls Derived::foo()
}
\end{lstlisting}

这种技术可以提供一种方式来设计类模板,该模板既可用于实例化具体类,也可用于使用继承进行扩展。然而,在一些成员函数上使用虚拟性来获得一个类还不够,这个类可以作为更好的基类来实现更特化的功能。这种开发方法需要更基本的设计决策,因此设计两个不同的工具(类或类模板架构)通常比将全部集成到一个模板结构中更为实用。






