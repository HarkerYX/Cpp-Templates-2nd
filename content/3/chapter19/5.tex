由于细节原因,基于SFINAE特征的一般方法在实践中可能会变得复杂。可以通过定义特征来说明这一点,该特征可以确定类型是否可以转换——若期望是某个基类或其派生类。IsConvertibleT特征决定是否可以将传递的第一个类型转换为传递的第二个类型:

\hspace*{\fill} \\ %插入空行
\noindent
\textit{traits/isconvertible.hpp}
\begin{lstlisting}[style=styleCXX]
#include <type_traits> // for true_type and false_type
#include <utility> // for declval

template<typename FROM, typename TO>
struct IsConvertibleHelper {
	private:
	// test() trying to call the helper aux(TO) for a FROM passed as F:
		static void aux(TO);
	template<typename F, typename T,
	typename = decltype(aux(std::declval<F>()))>
		static std::true_type test(void*);
	// test() fallback:
	template<typename, typename>
		static std::false_type test(...);
	public:
	using Type = decltype(test<FROM>(nullptr));
};

template<typename FROM, typename TO>
struct IsConvertibleT : IsConvertibleHelper<FROM, TO>::Type {
};

template<typename FROM, typename TO>
using IsConvertible = typename IsConvertibleT<FROM, TO>::Type;

template<typename FROM, typename TO>
constexpr bool isConvertible = IsConvertibleT<FROM, TO>::value;
\end{lstlisting}

使用函数重载的方法,如第19.4.1节。在辅助类内部,声明了两个名为test()的重载函数模板,其具有不同的返回类型,并为结果特征的基类声明了一个Type成员:

\begin{lstlisting}[style=styleCXX]
template<...> static std::true_type test(void*);
template<...> static std::false_type test(...);
...
using Type = decltype(test<FROM>(nullptr));
...
template<typename FROM, typename TO>
struct IsConvertibleT : IsConvertibleHelper<FROM, TO>::Type {
};
\end{lstlisting}

与往常一样,第一个test()重载设计在检查成功时匹配,而第二个重载会回退。因此,目标是使第一个test()重载在当类型FROM转换为类型TO时有效。为了实现这一点,再次给出test的第一个声明,一个虚拟的(未命名的)模板参数初始化使用了构造,当转换有效时该构造才有效。这个模板参数不能推导,也不会提供显式的模板参数。因此,参数将进行替换,若替换失败,将丢弃test()的声明。

下面的操作无效:

\begin{lstlisting}[style=styleCXX]
static void aux(TO);
template<typename = decltype(aux(std::declval<FROM>()))>
	static char test(void*);
\end{lstlisting}

当解析成员函数模时,FROM和TO是确定的,因此一对转换无效的类型(例如,double*和int*)将在调用test()前立即触发错误(因此在SFINAE之外)。

因此,引入F作为一个特定的成员函数模板参数

\begin{lstlisting}[style=styleCXX]
static void aux(TO);
template<typename F, typename = decltype(aux(std::declval<F>()))>
	static char test(void*);
\end{lstlisting}

并在调用test()时显式地提供FROM类型作为模板参数,该参数出现在value的初始化中:

\begin{lstlisting}[style=styleCXX]
static constexpr bool value
	= isSame<decltype(test<FROM>(nullptr)), char>;
\end{lstlisting}

不使用构造函数,如何使用std::declval来生成一个值?若该值可转换为TO,则对aux()的调用有效,并且test()的声明匹配。否则,SFINAE将失败,并且只匹配回退声明。

因此,可以这样使用该特征:

\begin{lstlisting}[style=styleCXX]
IsConvertibleT<int, int>::value // yields true
IsConvertibleT<int, std::string>::value // yields false
IsConvertibleT<char const*, std::string>::value // yields true
IsConvertibleT<std::string, char const*>::value // yields false
\end{lstlisting}

\hspace*{\fill} \\ %插入空行
\noindent
\textbf{处理特殊情况}

IsConvertibleT还未正确处理的三种情况:

\begin{enumerate}
\item
向数组类型的转换应该生成false,但aux()声明中的TO类型参数会衰变为指针类型,因此对某些FROM类型为true。

\item
函数类型的转换应该生成false,但就像数组一样,实现只将它们视为衰变类型。

\item
转换为(const/volatile限定的)void类型应该生成true。因为参数类型不能是void类型(而且aux()是用这样参数参声明的),所以上面的实现没有成功实例化TO是void类型的情况。
\end{enumerate}

对于所有这些情况,需要额外的偏特化。为每一种可能的const和volatile限定符组合添加这样的特化其实很难。不过,可以向辅助类模板添加额外的模板参数,如下所示:

\begin{lstlisting}[style=styleCXX]
template<typename FROM, typename TO, bool = IsVoidT<TO>::value
								|| IsArrayT<TO>::value
								|| IsFunctionT<TO>::value>
struct IsConvertibleHelper {
	using Type = std::integral_constant<bool,
	IsVoidT<TO>::value
	&& IsVoidT<FROM>::value>;
};

template<typename FROM, typename TO>
struct IsConvertibleHelper<FROM,TO,false> {
	... // previous implementation of IsConvertibleHelper here
};
\end{lstlisting}

尔模板参数,对所有这些特殊情况使用主辅助特征的实现。若转换为数组或函数(因为IsVoidT<TO>是假的),或者FROM是void而TO不是,则生成false\_type;对于两个void类型,也将生成false\_type。其他情况都为第三个参数产生false参数,因此采用偏特化,这与已经讨论过的实现相对应。

请参阅19.8.2节和19.8.3节来讨论如何实现IsArrayT。

C++标准库提供了相应的类型特征std::is\_convertible<>,在第D.3.3节中有描述。



