
初始特征示例说明,可以定义依赖于类型的行为。在C和C++中,定义的函数更具体地称为值函数:其接受一些值作为参数,并返回另一个值作为结果。通过模板,还可以定义类型函数:接受某种类型作为参数并生成类型或常量作为结果的函数。

sizeof是一个有用的内置类型函数,其返回一个常量,描述给定类型参数的大小(以字节为单位)。类模板也可以用作类型函数。类型函数的参数是模板参数,结果作为成员类型或成员常量提取。例如,sizeof操作符可以实现如下接口:

%\hspace*{\fill} \\ %插入空行
\noindent
\textit{traits/sizeof.cpp}
\begin{lstlisting}[style=styleCXX]
#include <cstddef>
#include <iostream>

template<typename T>
struct TypeSize {
	static std::size_t const value = sizeof(T);
};

int main()
{
	std::cout << "TypeSize<int>::value = "
	<< TypeSize<int>::value << ’\n’;
}
\end{lstlisting}

因为有内置的sizeof操作符可用,这似乎不是很有用,但请注意,TypeSize<T>是一个类型,因此可以作为类模板参数传递。或者,TypeSize是一个模板,可以作为模板参数传递。

接下来的内容中,将开发更多的通用类型函数,并可以以这种方式作为特征类。

\subsubsubsection{19.3.1\hspace{0.2cm}元素类型}

假设有许多容器模板,例如std::vector<>和std::list<>,以及内置数组。需要一个类型函数,给定这样的容器类型,该类型函数生成元素类型。这可以通过使用偏特化来实现:

\hspace*{\fill} \\ %插入空行
\noindent
\textit{traits/elementtype.hpp}
\begin{lstlisting}[style=styleCXX]
#include <vector>
#include <list>

template<typename T>
struct ElementT; // primary template

template<typename T>
struct ElementT<std::vector<T>> { // partial specialization for std::vector
	using Type = T;
};

template<typename T>
struct ElementT<std::list<T>> { // partial specialization for std::list
	using Type = T;
};

...
template<typename T, std::size_t N>
struct ElementT<T[N]> { // partial specialization for arrays of known bounds
	using Type = T;
};

template<typename T>
struct ElementT<T[]> { // partial specialization for arrays of unknown bounds
	using Type = T;
};
...
\end{lstlisting}

注意,应该为所有可能的数组类型提供偏特化(详见第5.4节)。

可以这样使用类型函数:

%\hspace*{\fill} \\ %插入空行
\noindent
\textit{traits/elementtype.cpp}
\begin{lstlisting}[style=styleCXX]
#include "elementtype.hpp"
#include <vector>
#include <iostream>
#include <typeinfo>

template<typename T>
void printElementType (T const& c)
{
	std::cout << "Container of "
	<< typeid(typename ElementT<T>::Type).name()
	<< " elements.\n";
}

int main()
{
	std::vector<bool> s;
	printElementType(s);
	int arr[42];
	printElementType(arr);
}
\end{lstlisting}

偏特化允许实现类型函数,而不需要容器知道类型。然而,类型函数是与适用类型一起设计的,并且可以简化实现。如果容器类型定义了一个成员类型value\_type(就像标准容器所做的那样),就可以这样:

\begin{lstlisting}[style=styleCXX]
template<typename C>
struct ElementT {
	using Type = typename C::value_type;
};
\end{lstlisting}

这可以是默认实现,并且不排除没有定义合适的成员类型value\_type的容器类型的特化。

尽管如此,通常建议为类模板类型参数提供成员类型定义,以便在泛型代码中更容易访问(就像标准容器模板那样):

\begin{lstlisting}[style=styleCXX]
template<typename T1, typename T2, ...>
class X {
	public:
	using ... = T1;
	using ... = T2;
	...
};
\end{lstlisting}

如何使用类型函数?其可以根据容器类型对模板进行参数化,而不需要元素类型和其他特征的参数。例如,

\begin{lstlisting}[style=styleCXX]
template<typename T, typename C>
T sumOfElements (C const& c);
\end{lstlisting}

需要sumOfElements<int>(list)这样的语法来显式指定元素类型,可以声明

\begin{lstlisting}[style=styleCXX]
template<typename C>
typename ElementT<C>::Type sumOfElements (C const& c);
\end{lstlisting}

其中元素类型由类型函数确定。

观察特征是如何作为现有类型的扩展实现的,甚至可以为基本类型和封闭库的类型定义这些类型函数。

因为用于访问给定容器类型C的特征(通常,类中可以收集多个特征),所以ElementT类型称为特征类。因此,特征类不局限于描述容器参数的特征,而是描述任何类型的“主要参数”。

方便起见,可以为类型函数创建别名模板。例如,可以引入

\begin{lstlisting}[style=styleCXX]
template<typename T>
using ElementType = typename ElementT<T>::Type;
\end{lstlisting}

这允许进一步简化上面的sumOfElements声明为

\begin{lstlisting}[style=styleCXX]
template<typename C>
ElementType<C> sumOfElements (C const& c);
\end{lstlisting}


\subsubsubsection{19.3.2\hspace{0.2cm}特征变换}

除了提供对主参数类型的特定方面的访问外,特征还可以对类型执行转换,例如添加或删除引用或const和volatile限定符。

\hspace*{\fill} \\ %插入空行
\noindent
\textbf{删除引用}

例如,可以实现一个RemoveReferenceT特征,将引用类型转换为的底层对象或函数类型,只保留非引用类型:

%\hspace*{\fill} \\ %插入空行
\noindent
\textit{traits/removereference.hpp}
\begin{lstlisting}[style=styleCXX]
template<typename T>
struct RemoveReferenceT {
	using Type = T;
};

template<typename T>
struct RemoveReferenceT<T&> {
	using Type = T;
};

template<typename T>
struct RemoveReferenceT<T&&> {
	using Type = T;
};
\end{lstlisting}

同样,便的别名模板可使使用方式更简单:

\begin{lstlisting}[style=styleCXX]
template<typename T>
using RemoveReference = typename RemoveReference<T>::Type;
\end{lstlisting}

当使用有时会产生引用类型的构造派生类型时,可以从类型中删除引用,例如在第15.6节中讨论的T\&\&类型的函数参数的特殊推导规则。

C++标准库提供了相应的类型特征std::remove\_reference<>,在第D.4节中描述。

\hspace*{\fill} \\ %插入空行
\noindent
\textbf{添加引用}

类似地,可以取一个已存在的类型,并从中创建一个左值或右值引用(以及通常方便的别名模板):

\hspace*{\fill} \\ %插入空行
\noindent
\textit{traits/addreference.hpp}
\begin{lstlisting}[style=styleCXX]
template<typename T>
struct AddLValueReferenceT {
	using Type = T&;
};

template<typename T>
using AddLValueReference = typename AddLValueReferenceT<T>::Type;

template<typename T>
struct AddRValueReferenceT {
	using Type = T&&;
};

template<typename T>
using AddRValueReference = typename AddRValueReferenceT<T>::Type;
\end{lstlisting}

引用折叠规则(第15.6节)适用于此。调用AddLValueReference<int\&\&>会产生类型int\&\&(因此不需要通过偏特化手动实现)。

如果保留AddLValueReferenceT和AddRValueReferenceT,不引入其特化,那么别名实际上可以简化为

\begin{lstlisting}[style=styleCXX]
template<typename T>
using AddLValueReferenceT = T&;

template<typename T>
using AddRValueReferenceT = T&&;
\end{lstlisting}

可以在不实例化类模板的情况下进行实例化(一个轻量级的过程)。然而,这是有风险的,因为很可能想为特殊情况特化这些模板。如上所述,不能使用void作为这些模板的模板参数。一些显式特化可以解决这个问题:

\begin{lstlisting}[style=styleCXX]
template<>
struct AddLValueReferenceT<void> {
	using Type = void;
};

template<>
struct AddLValueReferenceT<void const> {
	using Type = void const;
};

template<>
struct AddLValueReferenceT<void volatile> {
	using Type = void volatile;
};

template<>
struct AddLValueReferenceT<void const volatile> {
	using Type = void const volatile;
};
\end{lstlisting}

AddRValueReferenceT也类似。有了这些,就必须根据类模板来制定别名模板,以确保采用特化(因为别名模板不能特化)模板。

C++标准库提供了相应的类型特征std::add\_lvalue\_reference<>和std::add\_rvalue\_reference<>,在第D.4节中有描述。标准模板包括void类型的特化。

\hspace*{\fill} \\ %插入空行
\noindent
\textbf{删除限定符}

转换特征可以分解或引入任何类型的复合类型,而不仅仅是引用。如果存在const限定符,可以删除:

\hspace*{\fill} \\ %插入空行
\noindent
\textit{traits/removeconst.hpp}
\begin{lstlisting}[style=styleCXX]
template<typename T>
struct RemoveConstT {
	using Type = T;
};

template<typename T>
struct RemoveConstT<T const> {
	using Type = T;
};

template<typename T>
using RemoveConst = typename RemoveConstT<T>::Type;
\end{lstlisting}

此外,特征转换可以组合,比如创建一个RemoveCVT特征,可以同时删除const和volatile:

\hspace*{\fill} \\ %插入空行
\noindent
\textit{traits/removecv.hpp}
\begin{lstlisting}[style=styleCXX]
#include "removeconst.hpp"
#include "removevolatile.hpp"

template<typename T>
struct RemoveCVT : RemoveConstT<typename RemoveVolatileT<T>::Type> {
};

template<typename T>
using RemoveCV = typename RemoveCVT<T>::Type;
\end{lstlisting}

关于RemoveCVT的定义,有两点需要注意。首先,同时使用RemoveConstT和RemoveVolatileT,首先删除volatile(如果存在),然后将结果类型传递给RemoveConstT。

\begin{tcolorbox}[colback=webgreen!5!white,colframe=webgreen!75!black]
\hspace*{0.75cm}限定符删除的顺序没有语义上的影响:可以先删除volatile,然后再删除const。
\end{tcolorbox}

其次,使用元函数转发从RemoveConstT继承Type成员,而不是声明自己与RemoveConstT特化中相同的Type成员。这里使用元函数转发只是为了减少RemoveCVT定义中的输入量,但当没有为所有输入定义元函数时,元函数转发也很有用,这种技术将在第19.4节中进一步讨论。

别名模板RemoveCV可以简化为

\begin{lstlisting}[style=styleCXX]
template<typename T>
using RemoveCV = RemoveConst<RemoveVolatile<T>>;
\end{lstlisting}

同样,这只在RemoveCVT没有特化的情况下有效。与AddLValueReference和AddRValueReference不同,没有原因需要进行这样的特化。

C++标准库也提供了相应的类型特征std::remove\_volatile<>, std::remove\_const<>,和std::remove\_cv<>,在第D.4节中有描述。

\hspace*{\fill} \\ %插入空行
\noindent
\textbf{衰变}

为了完善对特征转换的讨论,我们开发了一个特征,按值向参数传递时模拟类型转换。派生自C语言,参数会产生衰变(将数组类型转换为指针,将函数类型转换为指向函数的指针类型;请参阅第7.4节和第11.1.1节),并删除所有顶级的const、volatile或引用限定符(因为解析函数调用时将忽略参数类型的类型限定符)。

这种按值传递的效果可以在下面的程序中看到,打印出编译器衰变指定类型后产生的实际参数类型:

\hspace*{\fill} \\ %插入空行
\noindent
\textit{traits/passbyvalue.cpp}
\begin{lstlisting}[style=styleCXX]
#include <iostream>
#include <typeinfo>
#include <type_traits>

template<typename T>
void f(T)
{ }

template<typename A>
void printParameterType(void (*)(A))
{
	std::cout << "Parameter type: " << typeid(A).name() << ’\n’;
	std::cout << "- is int: " << std::is_same<A,int>::value << ’\n’;
	std::cout << "- is const: " << std::is_const<A>::value << ’\n’;
	std::cout << "- is pointer: " << std::is_pointer<A>::value << ’\n’;
}

int main()
{
	printParameterType(&f<int>);
	printParameterType(&f<int const>);
	printParameterType(&f<int[7]>);
	printParameterType(&f<int(int)>);
}
\end{lstlisting}

程序的输出中,int参数保持不变,但是int const、int[7]和int(int)形参分别衰变为int、int*和int(*)(int)。

可以实现一个特征,产生相同的按值传递类型转换。为了与C++标准库特征std::decay相匹配,我们把它命名为DecayT。

\begin{tcolorbox}[colback=webgreen!5!white,colframe=webgreen!75!black]
\hspace*{0.75cm}使用术语衰变可能会有点混乱,因为在C中它只意味着从数组/函数类型到指针类型的转换,而在这里它还包括删除const/volatile限定符。
\end{tcolorbox}

首先,定义了非数组、非函数的情况,可以简单地删除const和volatile限定符:

\begin{lstlisting}[style=styleCXX]
template<typename T>
struct DecayT : RemoveCVT<T> {
};
\end{lstlisting}

接下来,处理数组到指针的衰变,这里要求使用偏特化来识别数组类型(有或没有绑定):

\begin{lstlisting}[style=styleCXX]
template<typename T>
struct DecayT<T[]> {
	using Type = T*;
};

template<typename T, std::size_t N>
struct DecayT<T[N]> {
	using Type = T*;
};
\end{lstlisting}

最后,处理函数到指针的衰变,必须匹配任何函数类型,而不管返回类型或参数类型的数量。为此,这里使用可变参数模板:

\begin{lstlisting}[style=styleCXX]
template<typename R, typename... Args>
struct DecayT<R(Args...)> {
	using Type = R (*)(Args...);
};

template<typename R, typename... Args>
struct DecayT<R(Args..., ...)> {
	using Type = R (*)(Args..., ...);
};
\end{lstlisting}

注意,第二个偏特化匹配任何使用C风格可变参数的函数类型。

\begin{tcolorbox}[colback=webgreen!5!white,colframe=webgreen!75!black]
\hspace*{0.75cm}严格地说,第二个省略号(…)之前的逗号可选,但这里为了清晰起见提供了逗号。由于省略可选,第一个偏特化中的函数类型在语法上不明确:可以解析为R(Args,…)(C风格的可变参数)或R(Args…(参数包)。之所以选择第二种解释,是因为Args是未扩展的参数包。需要其他解释的情况下,可以显式地添加逗号。
\end{tcolorbox}

主DecayT模板和它的四个偏特化一起实现参数类型衰变,如下例程序所示:

\hspace*{\fill} \\ %插入空行
\noindent
\textit{traits/decay.cpp}
\begin{lstlisting}[style=styleCXX]
#include <iostream>
#include <typeinfo>
#include <type_traits>
#include "decay.hpp"

template<typename T>
void printDecayedType()
{
	using A = typename DecayT<T>::Type;
	td::cout << "Parameter type: " << typeid(A).name() << ’\n’;
	std::cout << "- is int: " << std::is_same<A,int>::value << ’\n’;
	std::cout << "- is const: " << std::is_const<A>::value << ’\n’;
	std::cout << "- is pointer: " << std::is_pointer<A>::value << ’\n’;
}

int main()
{
	printDecayedType<int>();
	printDecayedType<int const>();
	printDecayedType<int[7]>();
	printDecayedType<int(int)>();
}
\end{lstlisting}

像往常一样,这里提供了一个别名模板:

\begin{lstlisting}[style=styleCXX]
template typename T>
using Decay = typename DecayT<T>::Type;
\end{lstlisting}

如上所示,C++标准库还提供了相应的类型特征std::decay<>,在第D.4节中描述。

\subsubsubsection{19.3.3\hspace{0.2cm}谓词特征}

我们已经研究和开发了单一类型的类型函数:给定一个类型,提供其他相关的类型或常量。但是,通常可以开发依赖多个参数的类型函数。这也导致了一种特殊形式的类型特征,类型谓词(产生布尔值的类型函数)。

\hspace*{\fill} \\ %插入空行
\noindent
\textbf{IsSameT}

IsSameT特征会产生两种类型是否相等:

\hspace*{\fill} \\ %插入空行
\noindent
\textit{traits/issame0.hpp}
\begin{lstlisting}[style=styleCXX]
template<typename T1, typename T2>
struct IsSameT {
	static constexpr bool value = false;
};

template<typename T>
struct IsSameT<T, T> {
	static constexpr bool value = true;
};
\end{lstlisting}

这里的主模板定义了两个不同的类型,通常作为模板参数传递。使得value成员为false。但是,使用偏特化,当传递的两个类型相同的特殊情况时,value为true。

例如,下面的表达式检查传递的模板参数是否为整数:

\begin{lstlisting}[style=styleCXX]
if (IsSameT<T, int>::value) ...
\end{lstlisting}

对于产生常量值的特征,不能提供别名模板,但可以提供constexpr变量模板来扮演相同的角色:

\begin{lstlisting}[style=styleCXX]
template<typename T1, typename T2>
constexpr bool isSame = IsSameT<T1, T2>::value;
\end{lstlisting}

C++标准库提供了相应的类型特征std::is\_same<>,在第D.3.3节中详述。

\hspace*{\fill} \\ %插入空行
\noindent
\textbf{true\_type和false\_type}

可以通过为两种结果(真和假)提供不同的类型来显著改进IsSameT的定义。如果声明一个类模板BoolConstant,其中包含TrueType和FalseType两种可能的实例化:

\hspace*{\fill} \\ %插入空行
\noindent
\textit{traits/boolconstant.hpp}
\begin{lstlisting}[style=styleCXX]
template<bool val>
struct BoolConstant {
	using Type = BoolConstant<val>;
	static constexpr bool value = val;
};
using TrueType = BoolConstant<true>;
using FalseType = BoolConstant<false>;
\end{lstlisting}

我们可以定义IsSameT,根据这两种类型是否匹配,其派生自TrueType或FalseType:

\hspace*{\fill} \\ %插入空行
\noindent
\textit{traits/issame.hpp}
\begin{lstlisting}[style=styleCXX]
#include "boolconstant.hpp"

template<typename T1, typename T2>
struct IsSameT : FalseType
{
};

template<typename T>
struct IsSameT<T, T> : TrueType
{
};
\end{lstlisting}

现在,结果类型为

\begin{lstlisting}[style=styleCXX]
IsSameT<T,int>
\end{lstlisting}

隐式转换为其基类TrueType或FalseType,不仅提供了相应的值成员,而且可以在编译时分配给不同的函数实现或类模板偏特化。例如:

\hspace*{\fill} \\ %插入空行
\noindent
\textit{traits/issame.cpp}
\begin{lstlisting}[style=styleCXX]
#include "issame.hpp"
#include <iostream>

template<typename T>
void fooImpl(T, TrueType)
{
	std::cout << "fooImpl(T,true) for int called\n";
}

template<typename T>
void fooImpl(T, FalseType)
{
	std::cout << "fooImpl(T,false) for other type called\n";
}

template<typename T>
void foo(T t)
{
	fooImpl(t, IsSameT<T,int>{}); // choose impl. depending on whether T is int
}

int main()
{
	foo(42); // calls fooImpl(42, TrueType)
	foo(7.7); // calls fooImpl(42, FalseType)
}
\end{lstlisting}

这种技术称为标签调度,将在第20.2节中介绍。

注意,BoolConstant的实现包含一个Type成员,其允许我们为IsSameT重新引入一个别名模板:

\begin{lstlisting}[style=styleCXX]
template<typename T>
using IsSame = typename IsSameT<T>::Type;
\end{lstlisting}

别名模板可以与变量模板isSame共存。

通常,产生布尔值的特征应该通过派生自TrueType和FalseType等类型来支持标记分配。为了泛型,应该只有一种类型表示真,一种类型表示假,而不是让每个泛型库为布尔常量定义自己的类型。

C++标准库在<type\_traits>中提供了相应的类型,因为C++11: std::true\_type和std::false\_type。</type\_traits>在C++11和C++14中,其定义如下:

\begin{lstlisting}[style=styleCXX]
namespace std {
	using true_type = integral_constant<bool, true>;
	using false_type = integral_constant<bool, false>;
}
\end{lstlisting}

C++17后,其定义为

\begin{lstlisting}[style=styleCXX]
namespace std {
	using true_type = bool_constant<true>;
	using false_type = bool_constant<false>;
}
\end{lstlisting}

其中bool\_constant在命名空间std中定义为

\begin{lstlisting}[style=styleCXX]
template<bool B>
using bool_constant = integral_constant<bool, B>;
\end{lstlisting}

请参阅第D.1.1节了解更多细节。

由于这个原因,本书后面章节将直接使用std::true\_type和std::false\_type,特别是在定义类型谓词时。

\subsubsubsection{19.3.4\hspace{0.2cm}结果类型特征} 

处理多种类型函数的另一个例子是结果类型特征,在编写操作符模板时非常有用。为了了解其用法,我们编写一个函数模板,可以添加两个数组容器:

\begin{lstlisting}[style=styleCXX]
template<typename T>
Array<T> operator+ (Array<T> const&, Array<T> const&);
\end{lstlisting}

看起来不错,但因为语言允许在int值上添加一个char值,所以更倾向于混合类型的数组操作。然后,需要确定结果模板的返回类型

\begin{lstlisting}[style=styleCXX]
template<typename T1, typename T2>
Array<???> operator+ (Array<T1> const&, Array<T2> const&);
\end{lstlisting}

除了在1.3节中介绍的不同方法外,结果类型模板可以在前面的声明中填写问号,如下所示:

\begin{lstlisting}[style=styleCXX]
template<typename T1, typename T2>
Array<typename PlusResultT<T1, T2>::Type>
operator+ (Array<T1> const&, Array<T2> const&);
\end{lstlisting}

或,若假设一个别名模板

\begin{lstlisting}[style=styleCXX]
template<typename T1, typename T2>
Array<PlusResult<T1, T2>>
operator+ (Array<T1> const&, Array<T2> const&);
\end{lstlisting}

PlusResultT特征通过加法操作符将两种(可能是不同的)类型的值相加,从而确定产生的类型:

\hspace*{\fill} \\ %插入空行
\noindent
\textit{traits/plus1.hpp}
\begin{lstlisting}[style=styleCXX]
template<typename T1, typename T2>
struct PlusResultT {
	using Type = decltype(T1() + T2());
};

template<typename T1, typename T2>
using PlusResult = typename PlusResultT<T1, T2>::Type;
\end{lstlisting}

这个特征模板使用decltype来计算表达式T1()+T2()的类型,将确定结果类型的工作(包括处理提升规则和重载操作符)交给编译器。

然而,对于我们的例子来说,decltype实际上保留了太多的信息(参见第15.10.2节关于decltype行为的描述)。PlusResultT可能会产生引用类型,但我们的Array类模板不是为处理引用类型而设计的。更实际地说,重载加法操作符可能返回const类类型的值:

\begin{lstlisting}[style=styleCXX]
class Integer { ... };
Integer const operator+ (Integer const&, Integer const&);
\end{lstlisting}

添加两个Array<Integer>值将得到一个Integer const数组,这很可能不是我们想要的。我们想要的是通过删除引用和限定符来转换结果类型,就像上一节那样:

\begin{lstlisting}[style=styleCXX]
template<typename T1, typename T2>
Array<RemoveCV<RemoveReference<PlusResult<T1, T2>>>>
operator+ (Array<T1> const&, Array<T2> const&);
\end{lstlisting}

这种特征嵌套在模板库中很常见,经常用于元编程上下文中。元编程将在第23章详细介绍。(别名模板对于像这样的多层嵌套特别有用;否则,需要在每一层添加typename和::Type后缀。)

当将两个元素类型(可能是不同的)数组相加时,数组加法操作符将正确地计算结果类型。但是,PlusResultT对元素类型T1和T2添加了一个限制:因为表达式T1()+T2()尝试对T1和T2类型的值进行初始化,这两种类型必须具有可访问的、非删除的默认构造函数(或者是非类类型)。Array类本身可能不需要元素类型的值初始化,因此这是一个不必要的限制。

\hspace*{\fill} \\ %插入空行
\noindent
\textbf{declval}

通过使用生成给定类型T值的函数,为加法表达式生成值很容易,不需要构造函数。为此,C++标准提供std::declval<>,在第11.2.3节中介绍。它在标准头文件<utility>中定义:

\begin{lstlisting}[style=styleCXX]
namespace std {
	template<typename T>
	add_rvalue_reference_t<T> declval() noexcept;
}
\end{lstlisting}

表达式declval<T>()生成T类型的值,不需要默认构造函数(或其他操作)。

这个函数模板故意没有定义,因其只用于decltype、sizeof或其他不需要定义的上下文中的操作。它还有另外两个有趣的属性:

\begin{itemize}
\item 
对于可引用的类型,返回类型是该类型的右值引用,这使得declval可以用于不能从函数返回的类型,如抽象类类型(具有纯虚函数的类)或数组类型。否则,从T到T\&\&的转换对作为表达式使用的declval<T>()的行为没有影响:两者都是右值(若T是一个对象类型),而左值引用类型由于引用折叠规则而不变(第15.6节中描述)。

\begin{tcolorbox}[colback=webgreen!5!white,colframe=webgreen!75!black]
\hspace*{0.75cm}通过直接使用decltype可以发现返回类型T和T\&\&之间的区别。然而,因为declval的使用限制,所以其没什么实际意义。
\end{tcolorbox}

\item 
noexcept异常说明说明declval不会导致表达式认为抛出异常。当在noexcept操作符的上下文中使用declval时,就可以使用了(第19.7.2节)。
\end{itemize}

使用declval,可以消除PlusResultT的值初始化要求:

\hspace*{\fill} \\ %插入空行
\noindent
\textit{traits/plus2.hpp}
\begin{lstlisting}[style=styleCXX]
#include <utility>

template<typename T1, typename T2>
struct PlusResultT {
	using Type = decltype(std::declval<T1>() + std::declval<T2>());
};

template<typename T1, typename T2>
using PlusResult = typename PlusResultT<T1, T2>::Type;
\end{lstlisting}

结果类型特征提供了一种确定特定操作的精确返回类型的方法,在描述函数模板的结果类型时很有用。




















