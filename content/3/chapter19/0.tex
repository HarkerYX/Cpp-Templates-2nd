模板能够对各种类型的类和函数进行参数化,引入尽可能多的模板参数以实现对类型或算法的每个方面的定制很有诱惑力。通过这种方式,“模板化”组件可以实例化,以满足外部代码的需求。然而,引入几十个模板参数来实现最大参数化是没必要的。必须在外部代码中指定所有相应的参数也非常繁琐,而且模板参数会使组件与其外部代码之间的关系复杂化。

我们引入的大多数参数都有默认值,参数可以由几个主要参数决定,这些参数可以省略。也可以给其他参数提供依赖于主要参数的默认值,这可以满足大多数情况,但默认值偶尔需要重写(对于特殊应用程序)。然而,其他参数与主要参数无关:除了存在几乎符合要求的默认值外,其本身就是主要参数。

特征(或特征模板)是C++编程工具,极大地促进了工业化模板设计中出现的那种额外的参数管理。本章中,将展示一些情况,证明这种模板是有用的,并演示各种技术,这些技术能够编写更健壮和强大的工具。

这里提供的大多数特性都可以在C++标准库中,以某种形式使用。然而,为了简单起见,我们经常给出简化的实现,其中省略了工业化实现(如标准库的那些)中的一些细节。出于这个原因,我们还使用了自己的命名方案,但这很容易与标准特征对应。