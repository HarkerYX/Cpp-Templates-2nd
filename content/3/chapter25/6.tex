原则上,可以定义operator[]来访问元组的元素。然而,与std::vector不同,元组的元素类型可以有不同,因此元组的operator[]必须是一个模板,其结果类型根据元素的索引不同而不同。这要求每个索引具有不同的类型,因此可以使用索引的类型来确定元素类型。

在第24.3节中介绍的类模板CTValue,可以在类型中编码数字索引,可以用其将operator[]定义为Tuple的成员:

\begin{lstlisting}[style=styleCXX]
template<typename T, T Index>
auto& operator[](CTValue<T, Index>) {
	return get<Index>(*this);
}
\end{lstlisting}

这里,在CTValue参数的类型中传递索引值来进行相应的get<>()调用。

可以这样使用这个类:

\begin{lstlisting}[style=styleCXX]
auto t = makeTuple(0, ’1’, 2.2f, std::string{"hello"});
auto a = t[CTValue<unsigned, 2>{}];
auto b = t[CTValue<unsigned, 3>{}];
\end{lstlisting}

a和b将由Tuple t中的第三和第四个值的类型和值初始化。

为了更方便地使用常量索引,可以使用constexpr实现字面操作符,从后缀为\_c的文字直接计算编译时的数字:

\hspace*{\fill} \\ %插入空行
\noindent
\textit{tuples/literals.hpp}
\begin{lstlisting}[style=styleCXX]
#include "ctvalue.hpp"
#include <cassert>
#include <cstddef>
// convert single char to corresponding int value at compile time:
constexpr int toInt(char c) {
	// hexadecimal letters:
	if (c >= ’A’ && c <= ’F’) {
		return static_cast<int>(c) - static_cast<int>(’A’) + 10;
	}
	if (c >= ’a’ && c <= ’f’) {
		return static_cast<int>(c) - static_cast<int>(’a’) + 10;
	}
	// other (disable ’.’ for floating-point literals):
	assert(c >= ’0’ && c <= ’9’);
	return static_cast<int>(c) - static_cast<int>(’0’);
}

// parse array of chars to corresponding int value at compile time:
template<std::size_t N>
constexpr int parseInt(char const (&arr)[N]) {
	int base = 10; // to handle base (default: decimal)
	int offset = 0; // to skip prefixes like 0x
	if (N > 2 && arr[0] == ’0’) {
		switch (arr[1]) {
			case ’x’: // prefix 0x or 0X, so hexadecimal
			case ’X’:
				base = 16;
				offset = 2;
				break;
			case ’b’: // prefix 0b or 0B (since C++14), so binary
			case ’B’:
				base = 2;
				offset = 2;
				break;
			default: // prefix 0, so octal
				base = 8;
				offset = 1;
				break;
		}
	}
	// iterate over all digits and compute resulting value:
	int value = 0;
	int multiplier = 1;
	for (std::size_t i = 0; i < N - offset; ++i) {
		if (arr[N-1-i] != ’\’’) { // ignore separating single quotes (e.g. in 1’000)
			value += toInt(arr[N-1-i]) * multiplier;
			multiplier *= base;
		}
	}
	return value;
}

// literal operator: parse integral literals with suffix _c as sequence of chars:
template<char... cs>
constexpr auto operator"" _c() {
	return CTValue<int, parseInt<sizeof...(cs)>({cs...})>{};
}
\end{lstlisting}

对于数字字面值,可以使用字面值操作符来推导出字面值的每个字符作为模板参数(详细信息请参阅第15.5.1节)。可以将字符传递给constexpr辅助函数parseInt(),该函数在编译时计算字符序列的值,并将其生成为CTValue。例如:

\begin{itemize}
\item 
42\_c生成CTValue<int,42>

\item 
0x815\_c生成CTValue<int,2069>

\item 
0b1111'1111\_c生成CTValue<int,255>

\begin{tcolorbox}[colback=webgreen!5!white,colframe=webgreen!75!black]
\hspace*{0.75cm}C++14开始,就支持二进制字面值的前缀0b和分隔数字的单引号字符。
\end{tcolorbox}
\end{itemize}

解析器不处理浮点字面值,断言会导致编译时错误,这是一个不能在编译时上下文中使用的运行时特性。

现在,就可以这样使用元组了:

\begin{lstlisting}[style=styleCXX]
auto t = makeTuple(0, ’1’, 2.2f, std::string{"hello"});
auto c = t[2_c];
auto d = t[3_c];
\end{lstlisting}

Boost.Hana(参见[boostana],适合对类型和值进行计算的元编程库)就使用了这种方法。



























