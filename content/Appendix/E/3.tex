
假设已经定义了概念IntegerLike<T>和StringLike<T>,并且决定编写模板来打印出这两个概念类型的值。可以这样做:

\begin{lstlisting}[style=styleCXX]
template<IntegerLike T> void print(T val); // #1
template<StringLike T> void print(T val); // #2
\end{lstlisting}

若没有不同的约束,这两个声明将声明相同的模板。但约束是模板签名的一部分,为了在重载解析期间区分模板。特别是,若发现两个模板都是可行的候选模板,但只有模板\#1满足其约束,那么重载会选择满足条件的模板。例如,假设int满足IntegerLike, std::string满足StringLike,但反之不行:

\begin{lstlisting}[style=styleCXX]
int main()
{
	printf(1); // selects template #1
	printf("1"s); // selects template #2
}
\end{lstlisting}

可以想象一个类似字符串的类型,支持类似整数的计算。若"6"\_NS和"7"\_NS是该类型的两个字面值,将这些字面值相乘将产生与"42"\_NS相同的值。这样的类型可能同时满足IntegerLike和StringLike,因此,像print("42"\_NS)这样的调用将有歧义。

\subsubsubsection{E.3.1\hspace{0.2cm}约束类型}

第一次讨论重载函数模板时,所涉及的约束通常互斥。例如,在使用IntegerLike和StringLike的示例中,可以设想同时满足这两个概念的类型,但希望这种情况比较罕见,以便重载打印模板仍然可用。

然而,有些概念从来不相互排斥,而是“包容”的。经典的例子是标准库的迭代器类别:输入迭代器、前向迭代器、双向迭代器、随机访问迭代器,以及C++17中的连续迭代器。

\begin{tcolorbox}[colback=webgreen!5!white,colframe=webgreen!75!black]
\hspace*{0.75cm}连续迭代器是C++17中引入的随机访问迭代器改进版。如果标签被更改,因为依赖于std::random\_access\_iterator\_tag的现有算法将不再选择,所以没有为它们添加std::continuous\_iterator\_tag。
\end{tcolorbox}

假设这里有一个ForwardIterator的定义:

\begin{lstlisting}[style=styleCXX]
template<typename T>
	concept ForwardIterator = ...;
\end{lstlisting}

那么“更精细的”概念BidirectionalIterator可以这样定义:

\begin{lstlisting}[style=styleCXX]
template<typename T>
	concept BidirectionIterator =
		ForwardIterator<T> &&
		requires (T it) {
			{ --it } -> T&
		};
\end{lstlisting}

在前向迭代器已经提供的功能基础上,添加加了前缀减法操作符的功能。

考虑std::advance()算法(我们将称之为advanceIter()),重载了使用受限模板的前向和双向迭代器:

\begin{lstlisting}[style=styleCXX]
template<ForwardIterator T, typename D>
void advanceIter(T& it, D n)
{
	assert(n >= 0);
	for (; n != 0; --n) { ++it; }
}

template<BidirectionalIterator T, typename D>
void advanceIter(T& it, D n)
{
	if (n > 0) {
		for (; n != 0; --n) { ++it; }
	} else if (n < 0) {
		for (; n != 0; ++n) { --it; }
	}
}
\end{lstlisting}

当使用普通的前向迭代器(即非双向迭代器)调用advanceIter()时,只有第一个模板的约束条件满足,重载解析很简单:选择第一个模板。然而,双向迭代器将满足两个模板的约束条件。这种情况下,当重载解析在其他方面并不偏好某个候选时,将偏好其约束包含其他候选约束的候选,而反之则不成立。包容的确切定义超出了这个介绍性附录的范畴,但只要知道约束C2<Ts...>定义为要求约束C1<Ts...>和其他约束(即\&\&),则前者包含后者,即可。

\begin{tcolorbox}[colback=webgreen!5!white,colframe=webgreen!75!black]
\hspace*{0.75cm}用于标准化的规范比这更强大。其将约束分解为“原子组件”的集合(包括require表达式的部分),并分析这些集合,看看其中一个是否是另一个的严格子集。
\end{tcolorbox}

我们的示例中,BidirectionalIterator<T>包含ForwardIterator<T>,因此使用双向迭代器调用时,首选第二个advanceIter()模板。

\subsubsubsection{E.3.2\hspace{0.2cm}约束和标签调度}

第20.2节中,讨论了使用标记调度重载advanceIter()算法的问题。该方法以一种相当优雅的方式集成到受限模板中。例如,输入迭代器和前向迭代器不能通过语法接口加以区分。所以,可以使用标签来定义其中之一:

\begin{lstlisting}[style=styleCXX]
template<typename T>
concept ForwardIterator =
InputIterator<T> &&
requires {
	typename std::iterator_traits<T>::iterator_category;
	is_convertible_v<std::iterator_traits<T>::iterator_category,
					std::forward_iterator_tag>;
};
\end{lstlisting}

这样,ForwardIterator<T>包含了InputIterator<T>,现在可以重载两个迭代器类别的约束模板。





















