Affectionately known as the ODR, the one-definition rule is a cornerstone for the well-formed structuring of C++ programs. The most common consequences of the ODR are simple enough to remember and apply: Define noninline functions or objects exactly once across all files, and define classes, inline functions, and inline variables at most once per translation unit, making sure that all definitions for the same entity are identical.

However, the devil is in the details, and when combined with template instantiation, these details can be daunting. This appendix is meant to provide a comprehensive overview of the ODR for the interested reader. We also indicate when specific related issues are expounded on in the main text.
















































