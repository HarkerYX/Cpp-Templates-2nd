
在本附录的介绍中所述,ODR有很多细节。我们根据规则的范围组织约束。

\subsubsubsection{A.3.1\hspace{0.2cm}程序的约束}

每个程序最多只能有以下的一个定义:

\begin{itemize}
\item 
非内联函数和非内联成员函数(包括函数模板的完全特化)

\item 
非内联变量(本质上是在命名空间作用域或全局作用域中声明的变量,没有静态说明符)

\item 
非内联静态数据成员
\end{itemize}

例如,由以下两个翻译单元组成的C++程序无效:

\begin{lstlisting}[style=styleCXX]
// translation unit 1:
int counter;

// translation unit 2:
int counter; // ERROR: defined twice (ODR violation)
\end{lstlisting}

此规则不适用于具有内部链接的实体(本质上是在全局作用域或命名空间作用域中使用静态说明符声明的实体),因为即使两个这样的实体具有相同的名称,也会认为是不同的。同样,在匿名命名空间中声明的实体如果出现在不同的翻译单元中,则认为是不同的;在C++11及以后的版本中,这些实体默认也具有内部链接,但在C++11之前,它们默认具有外部链接。例如,以下两个翻译单元可以组合成一个有效的C++程序:

\begin{lstlisting}[style=styleCXX]
// translation unit 1:
static int counter = 2; // unrelated to other translation units

namespace {
	void unique() // unrelated to other translation units
	{ }
}

// translation unit 1:
static int counter = 0; // unrelated to other translation units
namespace {
	void unique() // unrelated to other translation units
	{
		++counter;
	}
}

int main()
{
	unique();
}
\end{lstlisting}

此外,如果在conexpr if语句的丢弃分支之外的上下文中使用,则程序中必须只有前面提到的一项(该特性仅在C++17中可用;参见第14.6节)。在上下文中使用的术语具有确切的含义,表明在程序的某个地方存在对实体的某种引用,所以在直接的代码生成中需要实体。

\begin{tcolorbox}[colback=webgreen!5!white,colframe=webgreen!75!black]
\hspace*{0.75cm}各种优化技术可能会导致需要删除这一点,但该语言没有假设这种优化。
\end{tcolorbox}

此引用可以是对变量值的访问、对函数的调用或此类实体的地址。这个引用可以在源文件中显式引用,也可以隐式引用。例如,new表达式可以隐式地创建对关联的delete操作符的调用,以处理当构造函数抛出异常要求清理未使用(但已分配)内存时的情况。另一个例子包括复制构造函数,即使最终优化掉了,也必须定义(除非语言要求优化,C++17中经常出现这种情况)。虚函数也隐式使用(支持虚函数调用的内部结构使用),除非是纯虚函数。还有其他几种隐式用法,但为了简明起见,这里暂时忽略它们。

有些引用不构成使用:出现在未求值操作数中的引用(例如,sizeof或decltype操作符的操作数)。typeid操作符(参见第9.1.1节)的操作数只在某些情况下不计算。具体来说,如果引用作为typeid操作符的一部分出现,就不是使用了,除非typeid操作符的参数最终指定了一个多态对象(可能继承了虚函数的对象)。例如,考虑以下单文件程序:

\begin{lstlisting}[style=styleCXX]
#include <typeinfo>

class Decider {
	#if defined(DYNAMIC)
	virtual ~Decider() {
	}
	#endif
};

extern Decider d;

int main()
{
	char const* name = typeid(d).name();
	return (int)sizeof(d);
}
\end{lstlisting}

当预处理器符号DYNAMIC未定义时,这是一个有效的程序。实际上,变量d没有定义,但是sizeof(d)中对d的引用并不构成使用,而typeid(d)中的引用只有在d是多态类型对象时才构成使用(在运行时之前,不总确定多态类型id操作的结果)。

根据C++标准,本节描述的约束不需要从C++实现进行诊断。实践中,这些符号就是链接器报告为重复或缺失的定义。

\subsubsubsection{A.3.2\hspace{0.2cm}翻译单元的限制}

翻译单元中,实体都不能定义超过一次。所以下面的例子是无效的C++:

\begin{lstlisting}[style=styleCXX]
inline void f() {}
inline void f() {} // ERROR: duplicate definition
\end{lstlisting}

这也是在头文件中使用守卫包围代码的主要原因之一:

\begin{lstlisting}[style=styleCXX]
// guarddemo.hpp:
#ifndef GUARDDEMO_HPP
#define GUARDDEMO_HPP
...
#endif // GUARDDEMO_HPP
\end{lstlisting}

这样的保护确保头文件第二次\#include时,丢弃内容,从而避免其类、内联实体、模板等的重复定义。

ODR还指定必须在某些情况下定义某些实体。这可能是类类型、内联函数和内联变量的情况。接下来,我们将回顾详细的规则。

类类型X(包括结构体和联合体)必须在翻译单元中定义,然后才能在该翻译单元中进行下列类型的使用:

\begin{itemize}
\item 
创建类型为X的对象(例如,通过变量声明或通过new表达式)。当创建本身包含X类型对象的对象时,创建可以是间接的。

\item 
类型X的数据成员声明。

\item 
对X类型的对象应用sizeof或typeid操作符。

\item 
显式或隐式地访问类型X的成员。

\item 
使用类型的转换将表达式转换为或从类型X转换,或者使用隐式强制转换、static\_cast或dynamic\_cast换将表达式转换为指向X的指针或引用(void*除外)。

\item 
将值赋给X类型的对象。

\item 
定义或调用带有参数或返回类型为X的函数,但是,声明这样的函数并不需要定义其类型。
\end{itemize}

类型规则也适用于从类模板生成的类型X,需要在定义类型X的情况下定义相应的模板。这些情况会创建实例化点或POI(参见第14.3.2节)。

内联函数必须使用每个翻译单元中定义(这些翻译单元中调用它们或获取它们的地址)。然而,与类类型不同,它们的定义可以遵循以下使用要点:

\begin{lstlisting}[style=styleCXX]
inline int notSoFast();

int main()
{
	notSoFast();
}

inline int notSoFast()
{ }
\end{lstlisting}

虽然这是有效的C++代码,但一些基于旧技术的编译器实际上并不会“内联”调用尚未看到主体的函数;因此,预期的效果可能无法达到。

与类模板一样,使用由参数化函数声明(函数或成员函数模板,或类模板的成员函数)生成的函数会创建一个实例化点。然而,与类模板不同,相应的定义可以出现在实例化点之后。

本小节中解释的ODR的各个方面通常很容易通过C++编译器验证;因此,C++标准要求编译器在违反这些规则时发出某种诊断。异常是缺少参数化函数的定义,这种情况通常没有诊断信息。

\subsubsubsection{A.3.3\hspace{0.2cm}交叉翻译单元的等价约束}

多个翻译单元中定义某些类型可能会带来新的错误:多个定义不匹配。但传统的编译器技术每次只处理一个翻译单元,所以很难检测到这些错误。因此,C++标准并不要求检测或诊断多个定义之间的差异。然而,若违反交叉转换单元的约束,C++标准将其限定为未定义行为,从而任何合理或不合理的事情都可能发生。通常,这种未诊断的错误可能导致程序崩溃或错误的结果,但也可能导致其他更直接的损害(例如,文件损坏)。

\begin{tcolorbox}[colback=webgreen!5!white,colframe=webgreen!75!black]
\hspace*{0.75cm}gcc编译器的第1版实际上没有做到了这一点,只是某些情况下启动了肉鸽游戏。
\end{tcolorbox}

交叉翻译单元约束指定,实体在两个不同的地方定义时,这两个地方必须包含完全相同的标记序列(预处理之后保留的关键字、操作符、标识符等)。此外,这些标记必须在它们各自的上下文中表示相同的东西(例如,标识符可能需要引用相同的变量)。

考虑下面的例子:

\begin{lstlisting}[style=styleCXX]
// translation unit 1:
static int counter = 0;
inline void increaseCounter()
{
	++counter;
}

int main()
{ }

// translation unit 2:
static int counter = 0;
inline void increaseCounter()
{
	++counter;
}
\end{lstlisting}

这个例子是错误的,因为即使内联函数incrementounter()的标记序列在两个翻译单元中看起来相同,从而包含一个指向两个不同实体的标记计数器。不过,因为名为counter的两个变量具有内部链接(静态说明符),所以尽管名称相同,但它们不相关。注意,即使没有实际使用内联函数,这也是一个错误。

\#included头文件中可以放置在多个翻译单元中的实体定义,无论何时需要定义,都要包含这些定义,确保在几乎所有情况下标记序列都相同。

\begin{tcolorbox}[colback=webgreen!5!white,colframe=webgreen!75!black]
\hspace*{0.75cm}条件编译指令在不同的翻译单元中会有不同的计算结果,所以小心使用这些指令。其他更不常见差异也可能存在。
\end{tcolorbox}

使用这种方法,两个相同的标记引用不同事物的情况将变得相当罕见。当它发生时,产生的错误往往是隐秘的,难以查找。

交叉转换单元约束不仅可以在多个地方定义的实体,还适用于声明中的默认参数。换句话说,下面的程序有未定义行为:

\begin{lstlisting}[style=styleCXX]
// translation unit 1:
void unused(int = 3);
int main()
{ }

// translation unit 2:
void unused(int = 4);
\end{lstlisting}

这里,标记流的等价有时会涉及微妙的隐式影响。下面的例子是从C++标准中提取出来的(稍微修改了一下):

\begin{lstlisting}[style=styleCXX]
// translation unit 1:
class X {
	public:
	X(int, int);
	X(int, int, int);
};

X::X(int, int = 0)
{ }

class D {
	X x = 0;
};

D d1; // X(int, int) called by D()

// translation unit 2:
class X {
	public:
	X(int, int);
	X(int, int, int);
};

X::X(int, int = 0, int = 0)
{ }

class D : public X {
	X x = 0;
};

D d2; // X(int, int, int) called by D()
\end{lstlisting}

本例中,出现问题是因为类D隐式生成的默认构造函数在两个转换单元中不同。一个调用带两个参数的X构造函数,另一个调用带三个参数的X构造函数。如果有什么区别的话,就是将默认参数限制在程序中的一个位置(可能的话,这个位置应该在头文件中)的动机。幸运的是,在类外定义上放置默认参数是一种少见的做法。

对于相同的标记必须引用相同的实体规则,当然也有例外。如果相同的标记引用了具有相同值的不相关常量,并且没有使用结果表达式的地址(甚至通过将引用绑定到产生该常量的变量,也没有隐式地使用),那么这些标记等价。该异常允许的程序结构如下所示:

\begin{lstlisting}[style=styleCXX]
// header.hpp:
#ifndef HEADER_HPP
#define HEADER_HPP

int const length = 10;

class MiniBuffer {
	char buf[length];
	...
};

#endif // HEADER_HPP
\end{lstlisting}

原则上,当头文件包含在两个不同的翻译单元中时,将创建两个名为length的不同常量变量。因为在此上下文中,const意味着static。然而,这种常量变量通常用于定义编译时常量值,而不是运行时的特定存储位置。因此,若不强制存在这样一个存储位置(通过引用变量的地址),两个常量有相同的值就足够了。

最后,关于模板的说明。模板中的名称绑定分为两个阶段,非依赖名称在模板定义点绑定。对于这些规则,等价规则的处理类似于其他非模板定义。对于在实例化点绑定的名称,必须在该点应用等价规则,绑定必须等价。





