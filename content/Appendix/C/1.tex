重载解析只是函数调用完整处理的一部分,并不是每个函数调用的一部分。首先,通过函数指针和通过指向成员函数的指针进行的调用不受重载解析的影响,因为要调用的函数完全(在运行时)由指针决定。其次,类函数宏不能重载,因此不需要重载解析。

在高层次上,对命名函数的调用可以按以下方式处理:

\begin{itemize}
\item 
查找名称以形成初始重载集。

\item 
若有必要,这个集合会以各种方式进行调整(例如,发生模板参数推导和替换,这会导致丢弃一些函数模板候选)。

\item 
何与调用完全不匹配的候选(在考虑隐式转换和默认参数之后)都将从重载集中删除。这就产生了一组匹配的候选函数。

\item 
执行重载解析以找到最佳候选。如果有,则选中;否则,调用具有歧义。

\item 
选中的候选需要检查。例如,如果是一个已删除函数(即用=delete定义的函数)或一个不可访问的私有成员函数,则编译器会发出错误信息。
\end{itemize}

每个步骤都有自己的微妙之处,但是重载解析无疑是复杂的。幸运的是,一些简单的原则可以解释大多数情况。接下来我们将研究这些原则。














































